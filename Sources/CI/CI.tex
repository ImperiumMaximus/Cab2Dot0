\documentclass[a4paper,12pt,dvipsnames]{article}%
\usepackage{amsmath}%
\usepackage{amsfonts}%
\usepackage{amssymb}%
\usepackage{graphicx}
\usepackage{geometry}
\usepackage[hidelinks]{hyperref}
\usepackage{svg}
\usepackage{longtable}
\usepackage{array}
\usepackage{float}
\usepackage{pdflscape}
\usepackage{listings}
\usepackage[T1]{fontenc}
\usepackage[linesnumbered]{algorithm2e}
\usepackage[parfill]{parskip}
\usepackage{listings}
\usepackage[toc,page]{appendix}
\geometry{margin=1in}
\makeatletter\let\MPtrue\@minipagetrue\makeatother

\definecolor{javared}{rgb}{0.6,0,0} % for strings
\definecolor{javagreen}{rgb}{0.25,0.5,0.35} % comments
\definecolor{javapurple}{rgb}{0.5,0,0.35} % keywords
\definecolor{javadocblue}{rgb}{0.25,0.35,0.75} % javadoc

\begin{document}
% Nice Eclipse-style syntax highlight
\lstset{language=Java,
	basicstyle=\ttfamily,
	keywordstyle=\color{javapurple}\bfseries,
	stringstyle=\color{javared},
	commentstyle=\color{javagreen},
	morecomment=[s][\color{javadocblue}]{/**}{*/},
	numbers=left,
	numberstyle=\tiny\color{black},
	stepnumber=2,
	numbersep=10pt,
	tabsize=4,
	showspaces=false,
	showstringspaces=false} 
% \begin{lstlisting}[language=lang] if you want to override this
% example of how to use listing package (just write the code as you do in a normal text editor)
% \begin{lstlisting}
% public static void main(String[] args) {
%	System.out.println("Hello World!");
% }
% \end{lstlisting}
\begin{figure}
  \centering
	\def\svgwidth{\columnwidth}
    \resizebox{0.35\textwidth}{!}{\input{logo_polimi.pdf_tex}}
\end{figure}
\title{{\Huge \textbf{C}ode \textbf{I}nspection \textbf{D}ocument}\\{\Large Software Engineering 2}}

\author{Fioratto Raffaele, Longoni Nicol\`{o}
\\Politecnico di Milano
\\{\small A.Y. 2015/2016}
\\{\small Version 0.1}}
\date{December 10, 2015}
\maketitle
\newpage
\tableofcontents
\section{Introduction}
\subsection{Purpose}
This document represents the Design Document (DD). It describes completely the system in terms of components, analyzing their internal and external interaction (with the actors). It starts from the description of the system given in the Requirements Analysis and Specification Document. In this document the system is treated as a white box, i.e. the document ``opens" it and then dissects it starting from an high level overview to a more detailed representation. This document is addressed to all developers and programmers who have to implement the actual software.
\subsection{Scope}
\subsection{Definitions, Acronyms, Abbreviations}
\subsubsection{Definitions}
\subsubsection{Acronyms}
\begin{itemize}
	\item DD: Design Document
	\item RASD: Requirement Analysis and Specification Document
	\item ...
\end{itemize}
\subsubsection{Abbreviations}
\subsection{Reference Documents}
\begin{itemize}
	\item myTaxiService Specification Document
	\item myTaxiService RASD
	\item DD Structure Template
\end{itemize}
\subsection{Document Structure}
\begin{itemize}
	\item Section 1: Introduction, it gives a description of this document, some basic information in order to clearly understand the subsequent sections.
	\item Section 2: Architectural Design, it describes the software to be designed starting from an high level representation and then dissecting it providing a more in detail analysis. Diagrams are provided to better clarify this part.
	\item Section 3: Algorithm Design: it gives a description of the main algorithms that are implemented in the software, pseudo-code or flow diagrams are provided to describe them more clearly.
	\item Section 4: User Interface Design: starting from the mockups presented in the RASD, this part further explain the interaction between the actors and the system, and shows how it will look like.
	\item Section 5: Requirements Traceability, this part explain how the requirements defined in the RASD map into the design elements that we have defined in this document. All the requirements are covered.
	\item Section 6: References: provides a list of external documents and material that helped the writing of this document.
\end{itemize}
\newpage
\section{Classes assigned}
The following list includes a set of classes of the software's source code assigned to us and the packages in which they reside in.
Actually in our assignment, there's only one class to inspect and it is declared as follows:
\sourcesnippet{85-85}{\texttt{DeploymentDescriptorNode} declaration.}{Code/DeploymentDescriptorNode.java}
This class resides in the package declared at the beginning of the source Java file:
\sourcesnippet{41-41}{\texttt{DeploymentDescriptorNode} package membership declaration.}{Code/DeploymentDescriptorNode.java}
The package itself is inside the module called \texttt{Deployment Object Library}. \newline
The pathname of the file which this class resides in is\newline \texttt{appserver/deployment/dol/src/main/java/com/sun/enterprise/deployment/node}, and the filename for the source code is \texttt{DeploymentDescriptorNode.java}.
Below there's the diagram which describes the classes hierarchy for this class:
\begin{figure}[H]
	\centering
	\resizebox{2.5in}
	{!}{\input{ClassHierarchy.pdf_tex}}
	\caption{Class hierarchy for \texttt{DeploymentDescriptorNode}.}
\end{figure}
\newpage
\section{Functional Role}
\texttt{DeploymentDescriptorNode} is a base class responsible for handling a XML deployment descriptor, it implements the interface \texttt{XMLNode}.\newline
The XML Deployment Descriptor describes how a web application should be deployed. Each node reimplements this base class and overrides a bunch of methods for handling specific tag names in a XML file. 
For instance, we show here the code of a subclass of \texttt{DeploymentDescriptorNode} which is called \texttt{EjbReferenceNode}, it is responsible for handling \texttt{ejb-reference} XML node tags. We briefly analyze methods that are overridden. \newline
The first method is \texttt{getDescriptor()} which returns a descriptor of a XML Node being handled and encapsulates data regarding its XML node such as attributes and sub-tag names. 
\sourcesnippet{69-75}{\texttt{EjbReferenceNode.getDescriptor()} method implementation.}{Code/EjbReferenceNode.java}
Notice that, this method is implemented using Singleton design pattern.\newline
Second method is called \texttt{getDispatchTable()} which returns a \texttt{Map} associating each tag name to an appropriate function implemented in a \texttt{Descriptor} object returned by \texttt{getDescriptor()}. These functions will be called at runtime using reflection. With this approach, it is possible to achieve a high degree of flexibility since a parent class doesn't need to know anything about its children, they just supply to it how to handle at runtime each subtag and populate correctly its descriptor. \texttt{getDispatchTable()} method is implemented as follows:
\sourcesnippet{78-91}{\texttt{EjbReferenceNode.getDispatchTable()} method implementation.}{Code/EjbReferenceNode.java}
At last, \texttt{writeDescriptor()} method is responsible to perform the actual construction of a DOM tree node taking values from a \texttt{Descriptor} object whose attributes were populated previously. Eventually the DOM node is returned so that it will be inserted in the overall tree, this function is implemented by this code:
\sourcesnippet{93-127}{\texttt{EjbReferenceNode.writeDescriptor()} method implementation.}{Code/EjbReferenceNode.java}
As explained before, a subclass is responsible for writing a DOM representation of a handled tag, each tag is composed by attributes and possibly subtags. These last ones can be managed by a same class or delegated to other subclasses which are registered to the parent class. \newline This implies that all classes inheriting from \texttt{DeploymentDescriptorNode} are shaped like a tree with one root \texttt{DeploymentDescriptorNode} and they are arranged according to DTD. \newline SAX parser is responsible of interacting with correct subclasses depending on which tag is encountered while parsing a XML file by calling the method\newline\texttt{XMLNode.getHandlerFor(element)}. \texttt{element} parameter is an instance of \texttt{XMLElement} class which contains simple attributes to encapsulate tag names and (possible) XML namespaces. This function returns an object whose static type is \texttt{XMLNode} but its runtime type is one of possible subclasses of \newline \texttt{DeploymentDescriptorNode} which will handle a tag as described previously. \newline
Eventually a full DOM tree of the XML is created in memory which will be manipulated later by Glassfish in order to deploy the web application. \newline
This information has been gathered directly by the documentation of the source code written in Javadoc and it is placed within the code itself. Documentation is also publicly available from Glassfish official site. Another important source of information comes from Oracle, which points out in its documentation, what a XML Deployment Descriptor is and its role in the whole JEE ecosystem.
\newpage
\section{Issues found}
This section collects all the problems found by applying the checklist provided in the Code Inspection assignment document, only the rules violated are reported here, so we are assuming that if all the code inspected is consistent with the respect to a particular rule, it will be not listed here.
\newpage
\section{Other problems}
Although the checklist is composed by many rules, we believed that there are some checks that can be made in the code, in order to prevent software defects or misbehaviour. This section encloses additional issues that we think that are important to highlight so that when corrected, the overall code will be more robust and readable.
\subsection{Other problems related to \texttt{DeploymentDescriptorNode} class}
\begin{enumerate}
	\item \codeline{85}. Abstract classes should be named \texttt{AbstractXXX}.
	\sourcesnippet{85-85}{Class' name not starting with \texttt{Abstract}}{Code/DeploymentDescriptorNode.java}
	\item This class has to many methods, consider refactoring it.
	\item \codeline{94} and \codeline{100}. Avoid using implementation types like \texttt{``Hashtable''}; use the interface instead.
	\sourcesnippet{94-94}{Use of \texttt{``Hashtable''} instead of its base interface (1/2)}{Code/DeploymentDescriptorNode.java}
	\sourcesnippet{100-100}{Use of \texttt{``Hashtable''} instead of its base interface (2/2)}{Code/DeploymentDescriptorNode.java}
	\item The String literal \texttt{``enterprise.deployment.backend.addDescriptorFailure''} appears 5 times in this file (\codeline{152}, \codeline{194}, \codeline{198}, \codeline{202} and \codeline{204}); consider declaring it as a constant.
	\sourcesnippet{152-152}{Multiple usage of {\scriptsize\texttt{``enterprise.deployment.backend.addDescriptorFailure''}} (1/5)}{Code/DeploymentDescriptorNode.java}
	\sourcesnippet{194-194}{Multiple usage of {\scriptsize\texttt{``enterprise.deployment.backend.addDescriptorFailure''}} (2/5)}{Code/DeploymentDescriptorNode.java}
	\sourcesnippet{198-198}{Multiple usage of {\scriptsize\texttt{``enterprise.deployment.backend.addDescriptorFailure''}} (3/5)}{Code/DeploymentDescriptorNode.java}
	\sourcesnippet{202-202}{Multiple usage of {\scriptsize\texttt{``enterprise.deployment.backend.addDescriptorFailure''}} (4/5)}{Code/DeploymentDescriptorNode.java}
	\sourcesnippet{204-204}{Multiple usage of {\scriptsize\texttt{``enterprise.deployment.backend.addDescriptorFailure''}} (5/5)}{Code/DeploymentDescriptorNode.java}
	\item The String literal \texttt{``Error occurred''} appears 5 times in this file (\codeline{197}, \codeline{206}, \codeline{309}, \codeline{477} and \codeline{482}); consider declaring it as a constant.
	\sourcesnippet{197-197}{Multiple usage of \texttt{``Error occurred''} (1/5)}{Code/DeploymentDescriptorNode.java}
	\sourcesnippet{206-206}{Multiple usage of \texttt{``Error occurred''} (2/5)}{Code/DeploymentDescriptorNode.java}
	\sourcesnippet{309-309}{Multiple usage of \texttt{``Error occurred''} (3/5)}{Code/DeploymentDescriptorNode.java}
	\sourcesnippet{477-477}{Multiple usage of \texttt{``Error occurred''} (4/5)}{Code/DeploymentDescriptorNode.java}
	\sourcesnippet{482-482}{Multiple usage of \texttt{``Error occurred''} (5/5)}{Code/DeploymentDescriptorNode.java}
\end{enumerate}

\subsection{Other problems affecting method \texttt{handlesElement()}}
\begin{enumerate}
	\item \codeline{394} Add the \texttt{``@Override''} annotation above this method signature.
	\sourcesnippet{394-394}{Missing \texttt{``@Override''} annotation in \texttt{handlesElement()}}{Code/DeploymentDescriptorNode.java}
	\item \codeline{401} Consider using a local variable with the value returned by the method \break\texttt{element.getQName()} instead of chaining function calls.
	\sourcesnippet{401-401}{\texttt{element.getQName().equals(\dots)} methods chaining}{Code/DeploymentDescriptorNode.java}
	\item \codeline{399}. Consider declaring the variable using type \texttt{Enumeration<String>} in order to avoid casts in \codeline{400} and thus to improve readability.
	\sourcesnippet{399-400}{Add diamond operators to \texttt{``Enumeration''} with enclosing type \texttt{``String''} in variable declaration}{Code/DeploymentDescriptorNode.java}
\end{enumerate}

\subsection{Other problems related to method \texttt{setElementValue()}}
\begin{enumerate}
	\item \codeline{442} Add the \texttt{``@Override''} annotation above this method signature.
	\sourcesnippet{442-442}{Missing \texttt{``@Override''} annotation in \texttt{setElementValue()}}{Code/DeploymentDescriptorNode.java}
	\item \codeline{447} Merge this \texttt{if} statement with the enclosing one.
	\sourcesnippet{446-447}{Nested \texttt{if}s}{Code/DeploymentDescriptorNode.java}
	\item \codeline{450} and \codeline{457} Refactor this code to not nest more than 3 \texttt{if}/\texttt{for}/\texttt{while}/\texttt{switch}/\texttt{try} statements.
	\item \codeline{450}, \codeline{451}, \codeline{460}, \codeline{466}, \codeline{472}, \codeline{475}, \codeline{477}, \codeline{480}, \codeline{482} and \codeline{487}. Consider using a local variable with the value returned by the method \texttt{DOLUtils.getDefaultLogger()} instead of chaining function calls.
	\sourcesnippet{450-450}{\texttt{DOLUtils.getDefaultLogger().isLoggable(\dots)} methods chaining}{Code/DeploymentDescriptorNode.java} 
	\sourcesnippet{451-451}{\texttt{DOLUtils.getDefaultLogger().finer(\dots)} methods chaining}{Code/DeploymentDescriptorNode.java} 
	\sourcesnippet{460-460}{\texttt{DOLUtils.getDefaultLogger().log(\dots)} methods chaining (1/8)}{Code/DeploymentDescriptorNode.java} 
	\sourcesnippet{466-466}{\texttt{DOLUtils.getDefaultLogger().log(\dots)} methods chaining (2/8)}{Code/DeploymentDescriptorNode.java} 
	\sourcesnippet{472-472}{\texttt{DOLUtils.getDefaultLogger().log(\dots)} methods chaining (3/8)}{Code/DeploymentDescriptorNode.java}
	\sourcesnippet{475-475}{\texttt{DOLUtils.getDefaultLogger().log(\dots)} methods chaining (4/8)}{Code/DeploymentDescriptorNode.java} 
	\sourcesnippet{477-477}{\texttt{DOLUtils.getDefaultLogger().log(\dots)} methods chaining (5/8)}{Code/DeploymentDescriptorNode.java} 
	\sourcesnippet{480-480}{\texttt{DOLUtils.getDefaultLogger().log(\dots)} methods chaining (6/8)}{Code/DeploymentDescriptorNode.java} 
	\sourcesnippet{482-482}{\texttt{DOLUtils.getDefaultLogger().log(\dots)} methods chaining (7/8)}{Code/DeploymentDescriptorNode.java} 
	\sourcesnippet{487-487}{\texttt{DOLUtils.getDefaultLogger().log(\dots)} methods chaining (8/8)}{Code/DeploymentDescriptorNode.java} 
	\item \codeline{490}. Avoid unnecessary return statements.
	\sourcesnippet{490-490}{Unnecessary \texttt{return} statement at end of method}{Code/DeploymentDescriptorNode.java}
	\item \codeline{444}. Consider declaring the variable using type \texttt{Map<String, String>} in order to avoid casts in \codeline{458} and thus to improve readability.
	\sourcesnippet{444-444}{Add diamond operators to \texttt{``Map''} with enclosing type \texttt{``String, String''} in variable declaration}{Code/DeploymentDescriptorNode.java}
\end{enumerate}

\subsection{Other problems related to method \texttt{setDescriptorInfo()}}
\begin{enumerate}
	\item \codeline{516} and \codeline{517}. Consider using a local variable with the value returned by the method \texttt{DOLUtils.getDefaultLogger()} instead of chaining function calls.
	\sourcesnippet{516-516}{\texttt{DOLUtils.getDefaultLogger().isLoggable(\dots)} methods chaining}{Code/DeploymentDescriptorNode.java} 
	\sourcesnippet{517-517}{\texttt{DOLUtils.getDefaultLogger().finer(\dots)} methods chaining}{Code/DeploymentDescriptorNode.java} 
\end{enumerate}

\subsection{Other problems related to method \texttt{writeSubDescriptors()}}

\subsection{Other problems related to method \texttt{writeEjbReferenceDescriptors()}}
\newpage
\begin{appendices}
\section{Tools}
\begin{itemize}
	\item Document written in \LaTeX
	\item Basic MiKTeX 2.9.5721 64-bit -- \url{http://miktex.org/}
	\item Texmaker 4.5 -- \url{http://www.xm1math.net/texmaker/index.html}
	\item draw.io -- \url{https://www.draw.io/}
\end{itemize}
\break
\section{Hours of Work}
\begin{itemize}
	\item Raffaele Fioratto: 18 hours
	\item Nicol\`{o} Longoni: 18 hours
\end{itemize}
\break
\section{Code inspection checklist}
Below there is a checklist used to find violations in the code reported in the \nameref{sec:issues} section. It is provided as a reference to readers to better understand all reported problems.
\subsection*{Naming Conventions}\begin{enumerate}[label=C\arabic*., ref=C\arabic*]
\item \checklistref All class names, interface names, method names, class variables, method variables, and constants used should have meaningful names and do what the name suggests.
\item \checklistref If one-character variables are used, they are used only for temporary ``throwaway'' variables, such as those used in for loops.
\item \checklistref Class names are nouns, in mixed case, with the first letter of each word in capitalized. Examples: \texttt{class Raster}; \texttt{class ImageSprite};
\item \checklistref Interface names should be capitalized like classes.
\item \checklistref Method names should be verbs, with the first letter of each addition word capitalized. Examples: \texttt{getBackground()}; \texttt{computeTemperature()}.
\item \checklistref Class variables, also called attributes, are mixed case, but might begin with an underscore (`\texttt{\_}') followed by a lowercase first letter. All the remaining words in the variable name have their first letter capitalized. Examples: \texttt{\_windowHeight}, \texttt{timeSeriesData}.
\item \checklistref Constants are declared using all uppercase with words separated by an underscore. Examples: \texttt{MIN\_WIDTH}; \texttt{MAX\_HEIGHT}.
\end{enumerate}

\subsection*{Indention}\begin{enumerate}[resume, label=C\arabic*., ref=C\arabic*]
\item \checklistref Three or four spaces are used for indentation and done so consistently.
\item \checklistref No tabs are used to indent.
\end{enumerate}

\subsection*{Braces}\begin{enumerate}[resume, label=C\arabic*., ref=C\arabic*]
\item \checklistref Consistent bracing style is used, either the preferred ``Allman'' style (first brace goes underneath the opening block) or the ``Kernighan and Ritchie'' style (first brace is on the same line of the instruction that opens the new block).
\item \checklistref All \texttt{if}, \texttt{while}, \texttt{do-while}, \texttt{try-catch}, and \texttt{for} statements that have only one statement to execute are surrounded by curly braces. Example:
avoid this:

% this can be done way better by using LISTINGS package

\begin{verbatim}
    if ( condition )
        doThis();
\end{verbatim}

instead do this:

\begin{verbatim}
    if ( condition ) 
    {
        doThis(); 
    }
\end{verbatim}

\end{enumerate}

\subsection*{File Organization}\begin{enumerate}[resume, label=C\arabic*., ref=C\arabic*]
\item \checklistref Blank lines and optional comments are used to separate sections (beginning comments, package/import statements, class/interface declarations which include class variable/attributes declarations, constructors, and methods).
\item \checklistref Where practical, line length does not exceed 80 characters.
\item \checklistref When line length must exceed 80 characters, it does NOT exceed 120 characters.
\end{enumerate}

\subsection*{Wrapping Lines}\begin{enumerate}[resume, label=C\arabic*., ref=C\arabic*]
\item \checklistref Line break occurs after a comma or an operator.
\item \checklistref Higher-level breaks are used.
\item \checklistref A new statement is aligned with the beginning of the expression at the same level as the previous line.
\end{enumerate}

\subsection*{Comments}\begin{enumerate}[resume, label=C\arabic*., ref=C\arabic*]
\item \checklistref Comments are used to adequately explain what the class, interface, methods, and blocks of code are doing.
\item \checklistref Commented out code contains a reason for being commented out and a date it can be removed from the source file if determined it is no longer needed.
\end{enumerate}

\subsection*{Java Source Files}\begin{enumerate}[resume, label=C\arabic*., ref=C\arabic*]
\item \checklistref Each Java source file contains a single public class or interface.
\item \checklistref The public class is the first class or interface in the file.
\item \checklistref Check that the external program interfaces are implemented consistently with what is described in the javadoc.
\item \checklistref Check that the javadoc is complete (i.e., it covers all classes and files part of the set of classes assigned to you).
\end{enumerate}

\subsection*{Package and Import Statements}\begin{enumerate}[resume, label=C\arabic*., ref=C\arabic*]
\item \checklistref If any package statements are needed, they should be the first non-comment statements. Import statements follow.
\end{enumerate}

\subsection*{Class and Interface Declarations}\begin{enumerate}[resume, label=C\arabic*., ref=C\arabic*]
\item \checklistref The class or interface declarations shall be in the following order:
	\begin{enumerate}[label=\Alph*., ref=C25\Alph*]
		\item \label{C25A} class/interface documentation comment;
		\item \label{C25B} class or interface statement;
		\item \label{C25C} class/interface implementation comment, if necessary;
		\item \label{C25D}class (static) variables;
		\begin{enumerate}[label=\alph*.]
			\item first public class variables;
			\item next protected class variables;
			\item next package level (no access modifier);
			\item last private class variables.
		\end{enumerate}
		\item \label{C25E} instance variables;
		\begin{enumerate}[label=\alph*.]
			\item first public instance variables;
			\item next protected instance variables;
			\item next package level (no access modifier);
			\item last private instance variables.
		\end{enumerate}
		\item \label{C25F} constructors;
		\item \label{C25G} methods.
	\end{enumerate}
\item \checklistref Methods are grouped by functionality rather than by scope or accessibility.
\item \checklistref Check that the code is free of duplicates, long methods, big classes, breaking encapsulation, as well as if coupling and cohesion are adequate.
\end{enumerate}

\subsection*{Initialization and Declarations}\begin{enumerate}[resume, label=C\arabic*., ref=C\arabic*]
\item \checklistref Check that variables and class members are of the correct type. Check that they have the right visibility (public/private/protected).
\item \checklistref Check that variables are declared in the proper scope.
\item \checklistref Check that constructors are called when a new object is desired.
\item \checklistref Check that all object references are initialized before use.
\item \checklistref Variables are initialized where they are declared, unless dependent upon a computation.
\item \checklistref Declarations appear at the beginning of blocks (A block is any code surrounded by curly braces `\texttt{\{}' and `\texttt{\}}'). The exception is a variable can be declared in a \texttt{for} loop.
\end{enumerate}

\subsection*{Method Calls}\begin{enumerate}[resume, label=C\arabic*., ref=C\arabic*]
\item \checklistref Check that parameters are presented in the correct order.
\item \checklistref Check that the correct method is being called, or should it be a different method with a similar name.
\item \checklistref Check that method returned values are used properly.
\end{enumerate}

\subsection*{Arrays}\begin{enumerate}[resume, label=C\arabic*., ref=C\arabic*]
\item \checklistref Check that there are no off-by-one errors in array indexing (that is, all required array elements are correctly accessed through the index).
\item \checklistref Check that all array (or other collection) indexes have been prevented from going out-of-bounds.
\item \checklistref Check that constructors are called when a new array item is desired.
\end{enumerate}

\subsection*{Object Comparison}\begin{enumerate}[resume, label=C\arabic*., ref=C\arabic*]
\item \checklistref Check that all objects (including Strings) are compared with \texttt{equals} and not with \texttt{==}.
\end{enumerate}

\subsection*{Output Format}\begin{enumerate}[resume, label=C\arabic*., ref=C\arabic*]
\item \checklistref Check that displayed output is free of spelling and grammatical errors.
\item \checklistref Check that error messages are comprehensive and provide guidance as to how to correct the problem.
\item \checklistref Check that the output is formatted correctly in terms of line stepping and spacing.
\end{enumerate}

\subsection*{Computation, Comparisons and Assignments}\begin{enumerate}[resume, label=C\arabic*., ref=C\arabic*]
\item \checklistref Check that the implementation avoids ``brutish programming'': (see \url{http://users.csc.calpoly.edu/~jdalbey/SWE/CodeSmells/bonehead.html}). 
\item \checklistref Check order of computation/evaluation, operator precedence and parenthesizing.
\item \checklistref Check the liberal use of parenthesis is used to avoid operator precedence problems.
\item \checklistref Check that all denominators of a division are prevented from being zero.
\item \checklistref Check that integer arithmetic, especially division, are used appropriately to avoid causing unexpected truncation/rounding.
\item \checklistref Check that the comparison and Boolean operators are correct.
\item \checklistref Check throw-catch expressions, and check that the error condition is actually legitimate.
\item \checklistref Check that the code is free of any implicit type conversions.
\end{enumerate}

\subsection*{Exceptions}\begin{enumerate}[resume, label=C\arabic*., ref=C\arabic*]
\item \checklistref Check that the relevant exceptions are caught.
\item \checklistref Check that the appropriate action are taken for each catch block.
\end{enumerate}

\subsection*{Flow of Control}\begin{enumerate}[resume, label=C\arabic*., ref=C\arabic*]
\item \checklistref In a \texttt{switch} statement, check that all cases are addressed by break or return.
\item \checklistref Check that all switch statements have a default branch.
\item \checklistref Check that all loops are correctly formed, with the appropriate initialization, increment and termination expressions.
\end{enumerate}

\subsection*{Files}\begin{enumerate}[resume, label=C\arabic*., ref=C\arabic*]
\item \checklistref Check that all files are properly declared and opened.
\item \checklistref Check that all files are closed properly, even in the case of an error.
\item \checklistref Check that EOF conditions are detected and handled correctly.
\item \checklistref Check that all file exceptions are caught and dealt with accordingly.
\end{enumerate}



\end{appendices}
\end{document}