\documentclass[a4paper,12pt,dvipsnames]{article}%
\usepackage{amsmath}%
\usepackage{amsfonts}%
\usepackage{amssymb}%
\usepackage{graphicx}
\usepackage{geometry}
\usepackage[hidelinks]{hyperref}
\usepackage{svg}
\usepackage{longtable}
\usepackage{array}
\usepackage{float}
\usepackage{pdflscape}
\usepackage{listings}
\usepackage[T1]{fontenc}
\usepackage[linesnumbered]{algorithm2e}
\usepackage[parfill]{parskip}
\usepackage{listings}
\usepackage[toc,page]{appendix}
\geometry{margin=1in}
\makeatletter\let\MPtrue\@minipagetrue\makeatother

\definecolor{javared}{rgb}{0.6,0,0} % for strings
\definecolor{javagreen}{rgb}{0.25,0.5,0.35} % comments
\definecolor{javapurple}{rgb}{0.5,0,0.35} % keywords
\definecolor{javadocblue}{rgb}{0.25,0.35,0.75} % javadoc

\begin{document}
% Nice Eclipse-style syntax highlight
\lstset{language=Java,
	basicstyle=\ttfamily,
	keywordstyle=\color{javapurple}\bfseries,
	stringstyle=\color{javared},
	commentstyle=\color{javagreen},
	morecomment=[s][\color{javadocblue}]{/**}{*/},
	numbers=left,
	numberstyle=\tiny\color{black},
	stepnumber=2,
	numbersep=10pt,
	tabsize=4,
	showspaces=false,
	showstringspaces=false} 
% \begin{lstlisting}[language=lang] if you want to override this
% example of how to use listing package (just write the code as you do in a normal text editor)
% \begin{lstlisting}
% public static void main(String[] args) {
%	System.out.println("Hello World!");
% }
% \end{lstlisting}
\begin{figure}
  \centering
	\def\svgwidth{\columnwidth}
    \resizebox{0.35\textwidth}{!}{\input{logo_polimi.pdf_tex}}
\end{figure}
\title{{\Huge \textbf{D}esign \textbf{D}ocument}\\{\Large Software Engineering 2: ``myTaxiService''}}

\author{Fioratto Raffaele, Longoni Nicol\`{o}
\\Politecnico di Milano
\\{\small A.Y. 2015/2016}}
\date{November 14, 2015}
\maketitle
\newpage
\tableofcontents
\section{Introduction}
\subsection{Purpose}
This documentation represents the Design Document (DD). It describes completely the system in terms of components, analyzing their internal and external interactions (with the actors). It starts off with the description of the system given in the Requirements Analysis and Specification Document. In this document the system is treated like a white box, i.e. the document ``opens" it and then dissects it starting with a high level overview to a more detailed representation. This document is addressed to all developers and programmers who have to implement the actual software.
\subsection{Scope}
The aim of the project is to create a new brand system that optimizes an 
existing taxi service.
The system will be capable of automatizing and/or simplifying certain 
processes during requests or reservations of taxis.
It will also guarantee a fair management of taxi queues.
New passengers can sign up for service inserting some basic information in order to use service's features as soon as possible.
A passenger can request a taxi using web service or through mobile
application after registration. The system will be able to localize precisely the position
of the passenger, determining taxis that are available near
him/her. The system will select a taxi and then it will forward the request to one of its driver.
Upon confirmation, the system will notify the customer about the successful completion of the operation and the ETA of the taxi. A passenger can also reserve a taxi by specifying time and date, but they have to do it with at least two hours in advance. Cancellation is also permitted. The system will actually process the request ten minutes before the time specified during the reservation in the same way as described previously.
\subsection{Definitions, Acronyms, Abbreviations}
\subsubsection{Definitions}
\subsubsection{Acronyms}
\begin{itemize}
	\item DD: Design Document
	\item RASD: Requirement Analysis and Specification Document
	\item MVC: Model View Controller
	\item SOA: Service Oriented Application
	\item JEE: Java Enterprise Edition
	\item FSA: Finite State Automaton
\end{itemize}
\subsubsection{Abbreviations}
\subsection{Reference Documents}
\begin{itemize}
	\item myTaxiService Specification Document
	\item myTaxiService RASD
	\item DD Structure Template
\end{itemize}
\subsection{Document Structure}
\begin{itemize}
	\item Section 1: Introduction, it gives a description of this document, some basic information in order to clearly understand the subsequent sections.
	\item Section 2: Architectural Design, it describes the software to be designed starting with a high level representation and then dissecting it providing a more detailed analysis. Diagrams are provided to better clarify this part.
	\item Section 3: Algorithm Design: it gives a description of the main algorithms that are implemented in the software, pseudo-code or flow diagrams are provided to describe them in a more clear way.
	\item Section 4: User Interface Design: starting with mockups presented in the RASD, this part further explains interactions between actors and the system, and shows how it will look like.
	\item Section 5: Requirements Traceability: this part explains how requirements defined in the RASD map into the design elements that we shall define in this document.
	\item Section 6: References: this section provides a list of external documents and materials that allowed the composition of this documentation.
\end{itemize}
\break
\section{Classes assigned}
\newpage
\section{Functional Role}
\texttt{DeploymentDescriptorNode} is the base class responsible for handling the XML deployment descriptor, it implements the interface \texttt{XMLNode}.\newline
The XML Deployment Descriptor describes how a web application should be deployed. Each node re-implements this base class and overrides a bunch of methods for handling specific tag names in the XML file. 
As an example, we show here the code of a subclass of \texttt{DeploymentDescriptorNode} which is called \texttt{EjbReferenceNode} which is responsible for handling the \texttt{ejb-reference} XML node tag. We briefly analyze the methods that are overridden. \newline
The first method is \texttt{getDescriptor()} which returns the descriptor of the XML Node being handled and encapsulate all the data regarding the XML node itself such as attributes and sub-tag names. 
\sourcesnippet{69-75}{\texttt{EjbReferenceNode.getDescriptor()} method implementation.}{Code/EjbReferenceNode.java}
Notice that, the method is implemented using the Singleton design pattern.\newline
The second method is called \texttt{getDispatchTable()} which returns a \texttt{Map} associating each tag name to the appropriate function implemented in the \texttt{Descriptor} object returned by \texttt{getDescriptor()}. These functions will be called at runtime using reflection. With this approach one can achieve a high degree of flexibility given that a parent class doesn't need to know anything about the children, they just supply to it how to handle at runtime each sub tag and populate correctly the descriptor, the method \texttt{getDispatchTable()} is implemented as follows:
\sourcesnippet{78-91}{\texttt{EjbReferenceNode.getDispatchTable()} method implementation.}{Code/EjbReferenceNode.java}
At last, the method \texttt{writeDescriptor()} is responsible to perform the actual write of the DOM tree node taking the values from the \texttt{Descriptor} object whose attributes were populated previously. Eventually the DOM node is returned so that it will be inserted to the overall tree, the function is implemented by this code:
\sourcesnippet{93-127}{\texttt{EjbReferenceNode.writeDescriptor()} method implementation.}{Code/EjbReferenceNode.java}
As said before, a subclass is responsible for writing the DOM representation of the tag being handled, each tag is composed by attributes and possibly sub-tags. Sub-tags can be managed by the same class or delegated to other subclasses which are registered to the parent class. \newline This implies that all the classes inheriting from \texttt{DeploymentDescriptorNode} are shaped like a tree with one root \texttt{DeploymentDescriptorNode} and they are arranged according to the DTD. \newline The SAX parser is responsible of interacting with the correct subclass depending on which tag is encountered while parsing the XML file by calling the method\newline\texttt{XMLNode.getHandlerFor(element)}. The parameter \texttt{element} is an instance of the class \texttt{XMLElement}, which contains simple attributes for encapsulating the tag name and the (possible) XML namespace. The function returns an object whose static type is \texttt{XMLNode} but the runtime type is one of the possible subclasses of \newline \texttt{DeploymentDescriptorNode} which will handle the tag as described previously. \newline
Eventually the full DOM tree of the XML is created in memory which will be manipulated later by Glassfish in order to deploy the web application.
\newpage
\section{Issues found}
\label{sec:issues}
This section comprises all the problems found by applying the checklist provided in the Code Inspection assignment document, only the rules violated are reported here: we are assuming that if all the code inspected is consistent with the respect to a particular rule, it will be not listed here.
The issues are grouped by method.
\subsection{Issues related to \texttt{DeploymentDescriptorNode} class}
\begin{enumerate}
	\item \ref{C7} \codelines{120}{121}. The following class attribute is declared as \texttt{static final} and therefore its name should be in uppercase. \sourcesnippet{120-121}{\ref{C7} violation at \codelines{120}{121}}{Code/DeploymentDescriptorNode.java}
	\item \ref{C25D}  and \ref {C25E} \codelines{88}{121}. Instance and class variables are mixed together and also they are not grouped by scope visibility. \sourcesnippet{88-121}{\ref{C25D} and \ref{C25E} violations at \codelines{88}{121}}{Code/DeploymentDescriptorNode.java}
	\item \ref{C27} The overall class is 585 lines long and contains many methods, so it is better to split it in order to improve maintainability, and to increase the cohesion.
\end{enumerate}

\subsection{Issues affecting method \texttt{handlesElement()}}
This method begins at \codeline{394} and ends at \codeline{426}, below there's the list of violations found between this line range.
\begin{enumerate}
	\item \ref{C13} \codeline{399}. The line length exceeds 80 characters and can be broken at ``\texttt{;}'' just before the condition statement. \sourcesnippet{399-399}{\ref{C13} violation at \codeline{399}}{Code/DeploymentDescriptorNode.java}	
	\item \ref{C31} \codeline{401}, \codeline{403}, \codeline{411}, \codeline{416}, \codeline{419} and \codeline{423}. These statements doesn't properly check if either \texttt{element} or \texttt{element.getQName()} is not \texttt{null} before use either one of them or both.
	\sourcesnippet{401-401}{\ref{C31} violation at \codeline{401}}{Code/DeploymentDescriptorNode.java}
	\sourcesnippet{403-403}{\ref{C31} violation at \codeline{403}}{Code/DeploymentDescriptorNode.java}
	\sourcesnippet{411-411}{\ref{C31} violation at \codeline{411}}{Code/DeploymentDescriptorNode.java}
	\sourcesnippet{416-416}{\ref{C31} violation at \codeline{416}}{Code/DeploymentDescriptorNode.java}
	\sourcesnippet{419-419}{\ref{C31} violation at \codeline{419}}{Code/DeploymentDescriptorNode.java}
	\sourcesnippet{423-423}{\ref{C31} violation at \codeline{423}}{Code/DeploymentDescriptorNode.java}
	\item \ref{C33} \codeline{411}. Move the statement at the beginning of the block.
	\sourcesnippet{411-411}{\ref{C33} violation at \codeline{411}}{Code/DeploymentDescriptorNode.java}	
	\item \ref{C56} \codelines{399}{400}. The loop isn't well formed since the increment statement is inside its block of code rather than in the initialization. \sourcesnippet{399-400}{\ref{C56} violation at \codelines{399}{400}}{Code/DeploymentDescriptorNode.java}
\end{enumerate}

\subsection{Issues related to method \texttt{setElementValue()}}
This method begins at \codeline{442} and ends at \codeline{491}. Here's the list of violations found within this method:
\begin{enumerate}
	\item \ref{C2} \codeline{468}. Avoid using variables with short names.
	\sourcesnippet{468-468}{\ref{C2} violation at \codeline{468}}{Code/DeploymentDescriptorNode.java}
	\item \ref{C9} \codeline{443} and \codeline{468}. These line are indented using tabs, consider replacing them with spaces.
	\sourcesnippet{443-443}{\ref{C9} violation at \codeline{443}}{Code/DeploymentDescriptorNode.java}
	\sourcesnippet{468-468}{\ref{C9} violation at \codeline{468}}{Code/DeploymentDescriptorNode.java}
	\item \ref{C13} \codelines{470}{471}. Consider wrapping this comment in order to not exceed 80 characters on a single line.
	\sourcesnippet{470-471}{\ref{C13} violation at \codelines{470}{471}}{Code/DeploymentDescriptorNode.java}
	\item \ref{C13} \codeline{458}. Consider splitting this function call at ``\texttt{,}''.
	\sourcesnippet{458-458}{\ref{C13} violation at \codeline{458}}{Code/DeploymentDescriptorNode.java}
	\item \ref{C14} \codeline{443}. Commented statement length exceeds 120 characters, it is advisable to split it before the 3\textsuperscript{rd} ``\texttt{+}''.
	\sourcesnippet{443-443}{\ref{C14} violation at \codeline{443}}{Code/DeploymentDescriptorNode.java}
	\item \ref{C14} \codeline{451}. The string concatenation statement length inside the function call exceeds 120 characters, consider wrapping it at 1\textsuperscript{st} ``\texttt{+}''.
	\sourcesnippet{451-451}{\ref{C14} violation at \codeline{451}}{Code/DeploymentDescriptorNode.java}
	\item \ref{C19} \codeline{443}. Either remove this commented out statement or add a reason why it is commented and optionally a date when it can be removed.
	\sourcesnippet{443-443}{\ref{C19} violation at \codeline{443}}{Code/DeploymentDescriptorNode.java}
	\item \ref{C31} \codeline{447}, \codeline{448}, \codeline{458}, \codelines{460}{461}, \codelines{466}{467}, \codelines{472}{473} and \codelines{487}{488}. These statements doesn't properly check if either \texttt{element} or \texttt{element.getQName()} is not \texttt{null} before use either one of them or both.
	\sourcesnippet{447-447}{\ref{C31} violation at \codeline{447}}{Code/DeploymentDescriptorNode.java}
	\sourcesnippet{448-448}{\ref{C31} violation at \codeline{448}}{Code/DeploymentDescriptorNode.java}
	\sourcesnippet{458-458}{\ref{C31} violation at \codeline{458}}{Code/DeploymentDescriptorNode.java}
	\sourcesnippet{460-461}{\ref{C31} violation at \codelines{460}{461}}{Code/DeploymentDescriptorNode.java}
	\sourcesnippet{466-467}{\ref{C31} violation at \codelines{466}{467}}{Code/DeploymentDescriptorNode.java}
	\sourcesnippet{472-473}{\ref{C31} violation at \codelines{472}{473}}{Code/DeploymentDescriptorNode.java}
	\sourcesnippet{487-488}{\ref{C31} violation at \codelines{487}{488}}{Code/DeploymentDescriptorNode.java}
	\item \ref{C31} \codeline{458}, \codelines{460}{461}, \codelines{472}{473}, \codeline{486} and \codelines{487}{488}. These statements doesn't properly check if \texttt{value} is not \texttt{null} before use it.
	\sourcesnippet{458-458}{\ref{C31} violation at \codeline{458}}{Code/DeploymentDescriptorNode.java}
	\sourcesnippet{460-461}{\ref{C31} violation at \codelines{460}{461}}{Code/DeploymentDescriptorNode.java}
	\sourcesnippet{472-473}{\ref{C31} violation at \codelines{472}{473}}{Code/DeploymentDescriptorNode.java}
	\sourcesnippet{486-486}{\ref{C31} violation at \codeline{486}}{Code/DeploymentDescriptorNode.java}
	\sourcesnippet{487-488}{\ref{C31} violation at \codelines{487}{488}}{Code/DeploymentDescriptorNode.java}
	\item \ref{C31} \codeline{458}, \codelines{460}{461}, \codelines{472}{473}, \codeline{486} and \codelines{487}{488}. The following statements doesn't properly check if the value returned by the function \texttt{getDescriptor()} is not \texttt{null} before use it.
	\sourcesnippet{467-468}{\ref{C31} violation at \codelines{467}{468}}{Code/DeploymentDescriptorNode.java}
	\item \ref{C33} \codeline{468}. Move the statement at the beginning of the \texttt{catch} block.
	\sourcesnippet{465-468}{\ref{C33} violation at \codeline{468}}{Code/DeploymentDescriptorNode.java}
	\item \ref{C42} \codeline{477} and \codeline{482}. Generic \texttt{``Error occurred''} message, consider change it with a more explicative one.
	\sourcesnippet{477-477}{\ref{C42} violation at \codeline{477}}{Code/DeploymentDescriptorNode.java}
	\sourcesnippet{482-482}{\ref{C42} violation at \codeline{482}}{Code/DeploymentDescriptorNode.java}
	\item \ref{C44} \codeline{486}. Consider using the method \texttt{String.isEmpty()} which is already provided by the Java standard library.
	\sourcesnippet{486-486}{\ref{C44} violation at \codeline{486}}{Code/DeploymentDescriptorNode.java}
	\item \ref{C50} \codeline{479}. Catching a \texttt{Throwable} can lead to errors, consider use at least \texttt{Exception} or more specific subclasses, this can also improve code readability.  
	\sourcesnippet{479-479}{\ref{C50} violation at \codeline{479}}{Code/DeploymentDescriptorNode.java}
	\item \ref{C52} \codeline{447}, \codeline{448}, \codeline{458} and \codeline{467}. These statements should be surrounded with a proper \texttt{try-catch} block for the runtime exception \texttt{NullPointerException}. This exception is raised if the parameter of either the function \texttt{Map.containsKey()} or \texttt{Map.get()} is equal to \texttt{null}. Note that if this problem is fixed also the issues described at point 8 are automatically corrected. One can either check manually if the preconditions of these methods are met as described in point 8 or catching the relevant exception as pointed out in this issue.
	\sourcesnippet{447-447}{\ref{C52} violation at \codeline{447}}{Code/DeploymentDescriptorNode.java}
	\sourcesnippet{448-448}{\ref{C52} violation at \codeline{448}}{Code/DeploymentDescriptorNode.java}
	\sourcesnippet{458-458}{\ref{C52} violation at \codeline{458}}{Code/DeploymentDescriptorNode.java}
	\sourcesnippet{467-467}{\ref{C52} violation at \codeline{467}}{Code/DeploymentDescriptorNode.java}
\end{enumerate}

\subsection{Issues related to method \texttt{setDescriptorInfo()}}
This method begins at \codeline{512} and ends at \codeline{537} and the list of violations regarding this method is the following:
\begin{enumerate}
	\item \ref{C9} \codelines{519}{523}, \codelines{530}{531} and \codeline{536}. These lines are indented totally or partially using tabs.
	\sourcesnippet{519-523}{\ref{C9} violation at \codelines{519}{523}}{Code/DeploymentDescriptorNode.java}
	\sourcesnippet{530-531}{\ref{C9} violation at \codelines{530}{531}}{Code/DeploymentDescriptorNode.java}
	\sourcesnippet{536-536}{\ref{C9} violation at \codeline{536}}{Code/DeploymentDescriptorNode.java}
	\item \ref{C13} \codeline{521}, \codeline{526} and \codeline{533}. Consider splitting this function calls at ``\texttt{,}''. 
	\sourcesnippet{521-521}{\ref{C13} violation at \codeline{521}}{Code/DeploymentDescriptorNode.java}
	\sourcesnippet{526-526}{\ref{C13} violation at \codeline{526}}{Code/DeploymentDescriptorNode.java}
	\sourcesnippet{533-533}{\ref{C13} violation at \codeline{533}}{Code/DeploymentDescriptorNode.java}
	\item \ref{C14} \codeline{517}. The string concatenation statement length inside the function call exceeds 120 characters, consider wrapping it at 2\textsuperscript{nd} ``\texttt{+}''.
	\sourcesnippet{517-517}{\ref{C14} violation at \codeline{517}}{Code/DeploymentDescriptorNode.java}
	\item \ref{C31} \codeline{522}, \codeline{527} and \codeline{534}. These statements doesn't properly check if \texttt{value} is not \texttt{null} before use it.
	\sourcesnippet{522-522}{\ref{C31} violation at \codeline{522}}{Code/DeploymentDescriptorNode.java}
	\sourcesnippet{527-527}{\ref{C31} violation at \codeline{527}}{Code/DeploymentDescriptorNode.java}
	\sourcesnippet{534-534}{\ref{C31} violation at \codeline{534}}{Code/DeploymentDescriptorNode.java}
	\item \ref{C31} \codelines{521}{522}, \codelines{526}{527} and \codelines{533}{534}. These statements doesn't properly check if \texttt{target} is not \texttt{null} before use it.
	\sourcesnippet{521-522}{\ref{C31} violation at \codelines{521}{522}}{Code/DeploymentDescriptorNode.java}
	\sourcesnippet{526-527}{\ref{C31} violation at \codelines{526}{527}}{Code/DeploymentDescriptorNode.java}
	\sourcesnippet{533-534}{\ref{C31} violation at \codelines{533}{534}}{Code/DeploymentDescriptorNode.java}
	\item \ref{C31} \codeline{521}, \codeline{526} and \codeline{533}. These statements doesn't properly check if \texttt{methodName} is not \texttt{null} before use it.
	\sourcesnippet{521-521}{\ref{C31} violation at \codeline{521}}{Code/DeploymentDescriptorNode.java}
	\sourcesnippet{526-526}{\ref{C31} violation at \codeline{526}}{Code/DeploymentDescriptorNode.java}
	\sourcesnippet{533-533}{\ref{C31} violation at \codeline{533}}{Code/DeploymentDescriptorNode.java}
	\item \ref{C42} \codeline{517}. The message is misleading. It says that it's already in the target class' method but the invocation hasn't yet occurred. This could potentially lead to confusion if there are problems invoking the method at runtime using reflection, i.e. if one of the exceptions are caught.
	\sourcesnippet{517-517}{\ref{C42} violation at \codeline{517}}{Code/DeploymentDescriptorNode.java}
	\item \ref{C53} \codelines{524}{535}. Consider catching all the checked exceptions within the method rather than propagating the last one to the caller in case all the tries fail. If the error is unrecoverable a good solution is to throw a dedicated exception with a meaningful message containing the reason that raised the exceptional event. 
	\sourcesnippet{524-535}{\ref{C53} violation at \codelines{524}{535}}{Code/DeploymentDescriptorNode.java}
\end{enumerate}

\subsection{Issues related to method \texttt{writeSubDescriptors()}}
This method begins at line \codeline{626} and ends at line \codeline{664}. This is a list of issues found in it:
\begin{enumerate}
\item \ref{C1} \codeline{626}. Parameters passed \texttt{name}, \texttt{nodeName} and \texttt{descriptor} don't describe precisely their function in this method.
\sourcesnippet{626-626}{\ref{C1} violation at \codeline{626}}{Code/DeploymentDescriptorNode.java}
\item \ref{C13} \codeline{626}, \codeline{630} and \codeline{653}. These lines exceeded 80 characters, it can be inserted a break after \texttt{=} in last two cases. In first case it's suggested to begin a new line after a comma.
\sourcesnippet{626-626}{\ref{C13} violation at \codeline{626}}{Code/DeploymentDescriptorNode.java}
\sourcesnippet{630-630}{\ref{C13} violation at \codeline{630}}{Code/DeploymentDescriptorNode.java}\sourcesnippet{653-653}{\ref{C13} violation at \codeline{653}}{Code/DeploymentDescriptorNode.java}
\item \ref{C17} \codeline{639}. Wrong identation with respect to \codeline{637}. 
\sourcesnippet{637-639}{\ref{C17} violation at \codeline{639}}{Code/DeploymentDescriptorNode.java} 
\end{enumerate}
\subsection{Issues related to method \texttt{writeEjbReferenceDescriptors()}}
This method begins at line \codeline{799} and ends at line \codeline{}. Issues in this method are:
\begin{enumerate}
\item \ref{C11} \codelines{801}{802}. \texttt{If} followed by only one statement has to stay between curly brackets.
\sourcesnippet{801-802}{\ref{C11} violation at \codelines{801}{802}}{Code/DeploymentDescriptorNode.java}
\item \ref{C18} \codeline{805} and \codeline{815}. These comments aren't useful to explain their code's block.
\sourcesnippet{805-814}{\ref{C18} violation at \codelines{805}{814}}{Code/DeploymentDescriptorNode.java}
\sourcesnippet{815-819}{\ref{C18} violation at \codelines{815}{819}}{Code/DeploymentDescriptorNode.java}
\item \ref{C29} \codeline{804}. This attribute is used only to invoke \texttt{writeDescriptor} method, it is better if \texttt{subNode} is created when that method has to be invoked.
\sourcesnippet{804-804}{\ref{C29} violation at \codeline{804}}{Code/DeploymentDescriptorNode.java}
\item \ref{C40} \codeline{801}. Comparison has to be done with \texttt{equals()} method.
\sourcesnippet{801-801}{\ref{C40} violation at \codeline{801}}{Code/DeploymentDescriptorNode.java}
\item \ref{C56} \codelines{816}{829}. Looking this \texttt{for} cycle, it is better to iterate \texttt{localRefDescs} with a variable which will use directly as parameter in \texttt{writeDescriptor()} method.   
	\sourcesnippet{816-819}{\ref{C56} violation at \codelines{816}{819}}{Code/DeploymentDescriptorNode.java}
\end{enumerate}
\newpage
\section{Other problems}
Although the checklist is composed by many rules, we believed that there are some checks that can be made in the code, in order to prevent software defects or misbehaviour. This section encloses additional issues that we think that are important to highlight so that when corrected, the overall code will be more robust and readable.
\subsection{Other problems related to \texttt{DeploymentDescriptorNode} class}
\begin{enumerate}
	\item \codeline{85}. Abstract classes should be named \texttt{AbstractXXX}.
	\sourcesnippet{85-85}{Class' name not starting with \texttt{Abstract}}{Code/DeploymentDescriptorNode.java}
	\item This class has to many methods, consider refactoring it.
	\item \codeline{94} and \codeline{100}. Avoid using implementation types like \texttt{``Hashtable''}; use the interface instead.
	\sourcesnippet{94-94}{Use of \texttt{``Hashtable''} instead of its base interface (1/2)}{Code/DeploymentDescriptorNode.java}
	\sourcesnippet{100-100}{Use of \texttt{``Hashtable''} instead of its base interface (2/2)}{Code/DeploymentDescriptorNode.java}
	\item The String literal \texttt{``enterprise.deployment.backend.addDescriptorFailure''} appears 5 times in this file (\codeline{152}, \codeline{194}, \codeline{198}, \codeline{202} and \codeline{204}); consider declaring it as a constant.
	\sourcesnippet{152-152}{Multiple usage of {\scriptsize\texttt{``enterprise.deployment.backend.addDescriptorFailure''}} (1/5)}{Code/DeploymentDescriptorNode.java}
	\sourcesnippet{194-194}{Multiple usage of {\scriptsize\texttt{``enterprise.deployment.backend.addDescriptorFailure''}} (2/5)}{Code/DeploymentDescriptorNode.java}
	\sourcesnippet{198-198}{Multiple usage of {\scriptsize\texttt{``enterprise.deployment.backend.addDescriptorFailure''}} (3/5)}{Code/DeploymentDescriptorNode.java}
	\sourcesnippet{202-202}{Multiple usage of {\scriptsize\texttt{``enterprise.deployment.backend.addDescriptorFailure''}} (4/5)}{Code/DeploymentDescriptorNode.java}
	\sourcesnippet{204-204}{Multiple usage of {\scriptsize\texttt{``enterprise.deployment.backend.addDescriptorFailure''}} (5/5)}{Code/DeploymentDescriptorNode.java}
	\item The String literal \texttt{``Error occurred''} appears 5 times in this file (\codeline{197}, \codeline{206}, \codeline{309}, \codeline{477} and \codeline{482}); consider declaring it as a constant.
	\sourcesnippet{197-197}{Multiple usage of \texttt{``Error occurred''} (1/5)}{Code/DeploymentDescriptorNode.java}
	\sourcesnippet{206-206}{Multiple usage of \texttt{``Error occurred''} (2/5)}{Code/DeploymentDescriptorNode.java}
	\sourcesnippet{309-309}{Multiple usage of \texttt{``Error occurred''} (3/5)}{Code/DeploymentDescriptorNode.java}
	\sourcesnippet{477-477}{Multiple usage of \texttt{``Error occurred''} (4/5)}{Code/DeploymentDescriptorNode.java}
	\sourcesnippet{482-482}{Multiple usage of \texttt{``Error occurred''} (5/5)}{Code/DeploymentDescriptorNode.java}
\end{enumerate}

\subsection{Other problems affecting method \texttt{handlesElement()}}
\begin{enumerate}
	\item \codeline{394} Add the \texttt{``@Override''} annotation above this method signature.
	\sourcesnippet{394-394}{Missing \texttt{``@Override''} annotation in \texttt{handlesElement()}}{Code/DeploymentDescriptorNode.java}
	\item \codeline{401} Consider using a local variable with the value returned by the method \break\texttt{element.getQName()} instead of chaining function calls.
	\sourcesnippet{401-401}{\texttt{element.getQName().equals(\dots)} methods chaining}{Code/DeploymentDescriptorNode.java}
	\item \codeline{399}. Consider declaring the variable using type \texttt{Enumeration<String>} in order to avoid casts in \codeline{400} and thus to improve readability.
	\sourcesnippet{399-400}{Add diamond operators to \texttt{``Enumeration''} with enclosing type \texttt{``String''} in variable declaration}{Code/DeploymentDescriptorNode.java}
\end{enumerate}

\subsection{Other problems related to method \texttt{setElementValue()}}
\begin{enumerate}
	\item \codeline{442} Add the \texttt{``@Override''} annotation above this method signature.
	\sourcesnippet{442-442}{Missing \texttt{``@Override''} annotation in \texttt{setElementValue()}}{Code/DeploymentDescriptorNode.java}
	\item \codeline{447} Merge this \texttt{if} statement with the enclosing one.
	\sourcesnippet{446-447}{Nested \texttt{if}s}{Code/DeploymentDescriptorNode.java}
	\item \codeline{450} and \codeline{457} Refactor this code to not nest more than 3 \texttt{if}/\texttt{for}/\texttt{while}/\texttt{switch}/\texttt{try} statements.
	\item \codeline{450}, \codeline{451}, \codeline{460}, \codeline{466}, \codeline{472}, \codeline{475}, \codeline{477}, \codeline{480}, \codeline{482} and \codeline{487}. Consider using a local variable with the value returned by the method \texttt{DOLUtils.getDefaultLogger()} instead of chaining function calls.
	\sourcesnippet{450-450}{\texttt{DOLUtils.getDefaultLogger().isLoggable(\dots)} methods chaining}{Code/DeploymentDescriptorNode.java} 
	\sourcesnippet{451-451}{\texttt{DOLUtils.getDefaultLogger().finer(\dots)} methods chaining}{Code/DeploymentDescriptorNode.java} 
	\sourcesnippet{460-460}{\texttt{DOLUtils.getDefaultLogger().log(\dots)} methods chaining (1/8)}{Code/DeploymentDescriptorNode.java} 
	\sourcesnippet{466-466}{\texttt{DOLUtils.getDefaultLogger().log(\dots)} methods chaining (2/8)}{Code/DeploymentDescriptorNode.java} 
	\sourcesnippet{472-472}{\texttt{DOLUtils.getDefaultLogger().log(\dots)} methods chaining (3/8)}{Code/DeploymentDescriptorNode.java}
	\sourcesnippet{475-475}{\texttt{DOLUtils.getDefaultLogger().log(\dots)} methods chaining (4/8)}{Code/DeploymentDescriptorNode.java} 
	\sourcesnippet{477-477}{\texttt{DOLUtils.getDefaultLogger().log(\dots)} methods chaining (5/8)}{Code/DeploymentDescriptorNode.java} 
	\sourcesnippet{480-480}{\texttt{DOLUtils.getDefaultLogger().log(\dots)} methods chaining (6/8)}{Code/DeploymentDescriptorNode.java} 
	\sourcesnippet{482-482}{\texttt{DOLUtils.getDefaultLogger().log(\dots)} methods chaining (7/8)}{Code/DeploymentDescriptorNode.java} 
	\sourcesnippet{487-487}{\texttt{DOLUtils.getDefaultLogger().log(\dots)} methods chaining (8/8)}{Code/DeploymentDescriptorNode.java} 
	\item \codeline{490}. Avoid unnecessary return statements.
	\sourcesnippet{490-490}{Unnecessary \texttt{return} statement at end of method}{Code/DeploymentDescriptorNode.java}
	\item \codeline{444}. Consider declaring the variable using type \texttt{Map<String, String>} in order to avoid casts in \codeline{458} and thus to improve readability.
	\sourcesnippet{444-444}{Add diamond operators to \texttt{``Map''} with enclosing type \texttt{``String, String''} in variable declaration}{Code/DeploymentDescriptorNode.java}
	\item \codeline{458}. Useless cast to \texttt{String} of parameter \texttt{value} which is already declared with type \texttt{String}.
	\sourcesnippet{458-458}{Useless cast to \texttt{String} in variable declaration}{Code/DeploymentDescriptorNode.java}
\end{enumerate}

\subsection{Other problems related to method \texttt{setDescriptorInfo()}}
\begin{enumerate}
	\item \codeline{516} and \codeline{517}. Consider using a local variable with the value returned by the method \texttt{DOLUtils.getDefaultLogger()} instead of chaining function calls.
	\sourcesnippet{516-516}{\texttt{DOLUtils.getDefaultLogger().isLoggable(\dots)} methods chaining}{Code/DeploymentDescriptorNode.java} 
	\sourcesnippet{517-517}{\texttt{DOLUtils.getDefaultLogger().finer(\dots)} methods chaining}{Code/DeploymentDescriptorNode.java} 
\end{enumerate}

\subsection{Other problems related to method \texttt{writeSubDescriptors()}}

\subsection{Other problems related to method \texttt{writeEjbReferenceDescriptors()}}
\break
\begin{appendices}
\section{Tools}
\section{Hours of Work}
\end{appendices}
\end{document}