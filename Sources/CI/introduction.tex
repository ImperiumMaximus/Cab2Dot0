\newpage
\section{Introduction}
\subsection{Purpose}
Code Inspection document collects all results obtained by doing Code Inspection. The overall goal is to assess quality of a software by means of formal methods called Code Inspection techniques and more informal procedures. \newline
Probably the most important one is to apply a set of rules, which can be seen as a sort of checklist, to the code and to find whether it follows or not such rules. This procedure will be described more deeply in next sections and it is the key concept that will be used throughout the document. \newline
Doing Code Inspection has several benefits: to find coding mistakes is the most important and oversights that might be arisen during initial development phase in a systematic way. By reporting these issues in a document like this one, developers can acknowledge this kind of problems and then fix them. This will take to a more robust code and also it can potentially increase skills of developers themselves, so they hopefully will not repeating errors during following development phases in the same project or even in other ones, eventually this will reduce costs, the time needed to create a software in each of its phases and it will optimize the overall process.
\subsection{Scope}
As said before, Code Inspection is applied to an existing software project which is being developed. In this case, we are going to inspect ``Glassfish'': an open source implementation of a JEE server.
This project is written mainly in Java using Maven as a build tool, for this reason, its structure is highly modularized. \newline
Since there are a lot of contributors in this project, the entire code base is managed using a VCS, specifically SVN. A VCS is a centralized service that keeps track and provides control of all modifications made by contributors. Every time the source code of the software is modified, a new revision is uploaded to servers so that every contributor can check, discuss and improve changes.
Furthermore as long as servers are reliable, it is possible to eliminate local backups by developers.\newline
This document focuses on a particular version and revision of that software which is Glassfish v4.1.1 at revision 64219.
The entire code base consists of $\sim 1,000,000$ (Physical) SLoC\footnote{Code statistics generated using David A. Wheeler's ``SLOCCount''.}, which $\approx 84\%$ are written in Java and $\approx14\%$ in XML, the rest is written in other programming languages such as Python, Pascal and Ruby. Some important things to note are Development Effort Estimate which is $\approx 280$ Person-Years, (i.e. it requires almost 300 years to an average working person to develop this project on his own), and also Total Estimated Cost to Develop this software which is $\approx \$38.1$M, so it is trivial to understand that Glassfish is a very big piece of software and it is worth spending time and effort focusing on code inspection in order to improve quality and reliability of this software. 
\subsection{Acronyms and Abbreviations}
\subsubsection{Acronyms}
\begin{itemize}
	\item JEE: \textbf{J}ava \textbf{E}nterprise \textbf{E}dition
	\item VCS: \textbf{V}ersion \textbf{C}ontrol \textbf{S}ystem
	\item SVN: Apache \textbf{S}ub\textbf{V}ersio\textbf{N}
	\item API: \textbf{A}pplication \textbf{P}rogramming \textbf{I}nterface
	\item XML: e\textbf{X}tensible \textbf{M}arkup \textbf{L}anguage
	\item DTD: \textbf{D}ocument \textbf{T}ype \textbf{D}escription
	\item SAX: \textbf{S}imple \textbf{A}PI for \textbf{X}ML
	\item DOM: \textbf{D}ocument \textbf{O}bject \textbf{M}odel
	\item DOL: \textbf{D}eployment \textbf{O}bject \textbf{L}ibrary
	\item SLoC: \textbf{S}ource \textbf{L}ines of \textbf{C}ode
\end{itemize}
\subsubsection{Abbreviations}
\begin{itemize}
	\item Cn: n\textsuperscript{th} checklist element
	\item Ln: n\textsuperscript{th} line of code
	\item Li-j: lines of code in the interval i-j
\end{itemize}
\subsection{Reference Documents}
\begin{itemize}
	\item Code Inspection assignment document
	\item \href{http://glassfish.pompel.me/}{Glassfish Javadoc Documentation}\footnote{\texttt{http://glassfish.pompel.me/}}
	\item \href{http://docs.oracle.com/javaee/7/index.html}{Oracle JEE Documentation}\footnote{\texttt{http://docs.oracle.com/javaee/7/index.html}}
\end{itemize}
\subsection{Document Structure}
\begin{itemize}
	\item Section 1: Introduction, it gives a description of this document, some basic information in order to clearly understand subsequent sections.
	\item Section 2: Classes assigned, it briefly lists a set of classes that will be inspected throughout next sections.
	\item Section 3: Functional role, it describes what assigned classes do and how we determined this with the respect to some evidences such as Javadoc, diagrams and so on.
	\item Section 4: Issues found, it collects all problems found in analyzed code. For each item is stated which rules defined in the checklist mentioned before are violated and why. Snippets of violated code are also provided.
	\item Section 5: Other problems, it includes additional issues found during inspection that are not covered in the checklist, but worthy to be mentioned, so that potential software defects can be corrected.
	\item Section 6: Appendices, other extra information regarding this document. 
\end{itemize}
