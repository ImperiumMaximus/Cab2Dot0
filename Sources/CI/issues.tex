\newpage
\section{Issues found}
\label{sec:issues}
This section comprises all the problems found by applying the checklist provided in the Code Inspection assignment document, only the rules violated are reported here: we are assuming that if all the code inspected is consistent with the respect to a particular rule, it will be not listed here.
The issues are grouped by method.
\subsection{Issues related to \texttt{DeploymentDescriptorNode} class}
\begin{enumerate}
	\item \ref{C7} \codelines{120}{121}. The following class attribute is declared as \texttt{static final} and therefore its name should be in uppercase. \sourcesnippet{120-121}{\ref{C7} violation at \codelines{120}{121}}{Code/DeploymentDescriptorNode.java}
	\item \ref{C25D}  and \ref {C25E} \codelines{88}{121}. Instance and class variables are mixed together and also they are not grouped by scope visibility. \sourcesnippet{88-121}{\ref{C25D} and \ref{C25E} violations at \codelines{88}{121}}{Code/DeploymentDescriptorNode.java}
	\item \ref{C27} The overall class is 585 lines long and contains many methods, so it is better to split it in order to improve maintainability, and to increase the cohesion.
\end{enumerate}

\subsection{Issues affecting method \texttt{handlesElement()}}
This method begins at \codeline{394} and ends at \codeline{426}, below there's the list of violations found between this line range.
\begin{enumerate}
	\item \ref{C13} \codeline{399}. The line length exceeds 80 characters and can be broken at ``\texttt{;}'' just before the condition statement. \sourcesnippet{399-399}{\ref{C13} violation at \codeline{399}}{Code/DeploymentDescriptorNode.java}	
	\item \ref{C31} \codeline{401}, \codeline{403}, \codeline{411}, \codeline{416}, \codeline{419} and \codeline{423}. These statements doesn't properly check if either \texttt{element} or \texttt{element.getQName()} is not \texttt{null} before use either one of them or both.
	\sourcesnippet{401-401}{\ref{C31} violation at \codeline{401}}{Code/DeploymentDescriptorNode.java}
	\sourcesnippet{403-403}{\ref{C31} violation at \codeline{403}}{Code/DeploymentDescriptorNode.java}
	\sourcesnippet{411-411}{\ref{C31} violation at \codeline{411}}{Code/DeploymentDescriptorNode.java}
	\sourcesnippet{416-416}{\ref{C31} violation at \codeline{416}}{Code/DeploymentDescriptorNode.java}
	\sourcesnippet{419-419}{\ref{C31} violation at \codeline{419}}{Code/DeploymentDescriptorNode.java}
	\sourcesnippet{423-423}{\ref{C31} violation at \codeline{423}}{Code/DeploymentDescriptorNode.java}
	\item \ref{C33} \codeline{411}. Move the statement at the beginning of the block.
	\sourcesnippet{411-411}{\ref{C33} violation at \codeline{411}}{Code/DeploymentDescriptorNode.java}	
	\item \ref{C56} \codelines{399}{400}. The loop isn't well formed since the increment statement is inside its block of code rather than in the initialization. \sourcesnippet{399-400}{\ref{C56} violation at \codelines{399}{400}}{Code/DeploymentDescriptorNode.java}
\end{enumerate}

\subsection{Issues related to method \texttt{setElementValue()}}
This method begins at \codeline{442} and ends at \codeline{491}. Here's the list of violations found within this method:
\begin{enumerate}
	\item \ref{C2} \codeline{468}. Avoid using variables with short names.
	\sourcesnippet{468-468}{\ref{C2} violation at \codeline{468}}{Code/DeploymentDescriptorNode.java}
	\item \ref{C9} \codeline{443} and \codeline{468}. These line are indented using tabs, consider replacing them with spaces.
	\sourcesnippet{443-443}{\ref{C9} violation at \codeline{443}}{Code/DeploymentDescriptorNode.java}
	\sourcesnippet{468-468}{\ref{C9} violation at \codeline{468}}{Code/DeploymentDescriptorNode.java}
	\item \ref{C13} \codelines{470}{471}. Consider wrapping this comment in order to not exceed 80 characters on a single line.
	\sourcesnippet{470-471}{\ref{C13} violation at \codelines{470}{471}}{Code/DeploymentDescriptorNode.java}
	\item \ref{C13} \codeline{458}. Consider splitting this function call at ``\texttt{,}''.
	\sourcesnippet{458-458}{\ref{C13} violation at \codeline{458}}{Code/DeploymentDescriptorNode.java}
	\item \ref{C14} \codeline{443}. Commented statement length exceeds 120 characters, it is advisable to split it before the 3\textsuperscript{rd} ``\texttt{+}''.
	\sourcesnippet{443-443}{\ref{C14} violation at \codeline{443}}{Code/DeploymentDescriptorNode.java}
	\item \ref{C14} \codeline{451}. The string concatenation statement length inside the function call exceeds 120 characters, consider wrapping it at 1\textsuperscript{st} ``\texttt{+}''.
	\sourcesnippet{451-451}{\ref{C14} violation at \codeline{451}}{Code/DeploymentDescriptorNode.java}
	\item \ref{C19} \codeline{443}. Either remove this commented out statement or add a reason why it is commented and optionally a date when it can be removed.
	\sourcesnippet{443-443}{\ref{C19} violation at \codeline{443}}{Code/DeploymentDescriptorNode.java}
	\item \ref{C31} \codeline{447}, \codeline{448}, \codeline{458}, \codelines{460}{461}, \codelines{466}{467}, \codelines{472}{473} and \codelines{487}{488}. These statements doesn't properly check if either \texttt{element} or \texttt{element.getQName()} is not \texttt{null} before use either one of them or both.
	\sourcesnippet{447-447}{\ref{C31} violation at \codeline{447}}{Code/DeploymentDescriptorNode.java}
	\sourcesnippet{448-448}{\ref{C31} violation at \codeline{448}}{Code/DeploymentDescriptorNode.java}
	\sourcesnippet{458-458}{\ref{C31} violation at \codeline{458}}{Code/DeploymentDescriptorNode.java}
	\sourcesnippet{460-461}{\ref{C31} violation at \codelines{460}{461}}{Code/DeploymentDescriptorNode.java}
	\sourcesnippet{466-467}{\ref{C31} violation at \codelines{466}{467}}{Code/DeploymentDescriptorNode.java}
	\sourcesnippet{472-473}{\ref{C31} violation at \codelines{472}{473}}{Code/DeploymentDescriptorNode.java}
	\sourcesnippet{487-488}{\ref{C31} violation at \codelines{487}{488}}{Code/DeploymentDescriptorNode.java}
	\item \ref{C31} \codeline{458}, \codelines{460}{461}, \codelines{472}{473}, \codeline{486} and \codelines{487}{488}. These statements doesn't properly check if \texttt{value} is not \texttt{null} before use it.
	\sourcesnippet{458-458}{\ref{C31} violation at \codeline{458}}{Code/DeploymentDescriptorNode.java}
	\sourcesnippet{460-461}{\ref{C31} violation at \codelines{460}{461}}{Code/DeploymentDescriptorNode.java}
	\sourcesnippet{472-473}{\ref{C31} violation at \codelines{472}{473}}{Code/DeploymentDescriptorNode.java}
	\sourcesnippet{486-486}{\ref{C31} violation at \codeline{486}}{Code/DeploymentDescriptorNode.java}
	\sourcesnippet{487-488}{\ref{C31} violation at \codelines{487}{488}}{Code/DeploymentDescriptorNode.java}
	\item \ref{C31} \codeline{458}, \codelines{460}{461}, \codelines{472}{473}, \codeline{486} and \codelines{487}{488}. The following statements doesn't properly check if the value returned by the function \texttt{getDescriptor()} is not \texttt{null} before use it.
	\sourcesnippet{467-468}{\ref{C31} violation at \codelines{467}{468}}{Code/DeploymentDescriptorNode.java}
	\item \ref{C33} \codeline{468}. Move the statement at the beginning of the \texttt{catch} block.
	\sourcesnippet{465-468}{\ref{C33} violation at \codeline{468}}{Code/DeploymentDescriptorNode.java}
	\item \ref{C42} \codeline{477} and \codeline{482}. Generic \texttt{``Error occurred''} message, consider change it with a more explicative one.
	\sourcesnippet{477-477}{\ref{C42} violation at \codeline{477}}{Code/DeploymentDescriptorNode.java}
	\sourcesnippet{482-482}{\ref{C42} violation at \codeline{482}}{Code/DeploymentDescriptorNode.java}
	\item \ref{C44} \codeline{486}. Consider using the method \texttt{String.isEmpty()} which is already provided by the Java standard library.
	\sourcesnippet{486-486}{\ref{C44} violation at \codeline{486}}{Code/DeploymentDescriptorNode.java}
	\item \ref{C50} \codeline{479}. Catching a \texttt{Throwable} can lead to errors, consider use at least \texttt{Exception} or more specific subclasses, this can also improve code readability.  
	\sourcesnippet{479-479}{\ref{C50} violation at \codeline{479}}{Code/DeploymentDescriptorNode.java}
	\item \ref{C52} \codeline{447}, \codeline{448}, \codeline{458} and \codeline{467}. These statements should be surrounded with a proper \texttt{try-catch} block for the runtime exception \texttt{NullPointerException}. This exception is raised if the parameter of either the function \texttt{Map.containsKey()} or \texttt{Map.get()} is equal to \texttt{null}. Note that if this problem is fixed also the issues described at point 8 are automatically corrected. One can either check manually if the preconditions of these methods are met as described in point 8 or catching the relevant exception as pointed out in this issue.
	\sourcesnippet{447-447}{\ref{C52} violation at \codeline{447}}{Code/DeploymentDescriptorNode.java}
	\sourcesnippet{448-448}{\ref{C52} violation at \codeline{448}}{Code/DeploymentDescriptorNode.java}
	\sourcesnippet{458-458}{\ref{C52} violation at \codeline{458}}{Code/DeploymentDescriptorNode.java}
	\sourcesnippet{467-467}{\ref{C52} violation at \codeline{467}}{Code/DeploymentDescriptorNode.java}
\end{enumerate}

\subsection{Issues related to method \texttt{setDescriptorInfo()}}
This method begins at \codeline{512} and ends at \codeline{537} and the list of violations regarding this method is the following:
\begin{enumerate}
	\item \ref{C9} \codelines{519}{523}, \codelines{530}{531} and \codeline{536}. These lines are indented totally or partially using tabs.
	\sourcesnippet{519-523}{\ref{C9} violation at \codelines{519}{523}}{Code/DeploymentDescriptorNode.java}
	\sourcesnippet{530-531}{\ref{C9} violation at \codelines{530}{531}}{Code/DeploymentDescriptorNode.java}
	\sourcesnippet{536-536}{\ref{C9} violation at \codeline{536}}{Code/DeploymentDescriptorNode.java}
	\item \ref{C13} \codeline{521}, \codeline{526} and \codeline{533}. Consider splitting this function calls at ``\texttt{,}''. 
	\sourcesnippet{521-521}{\ref{C13} violation at \codeline{521}}{Code/DeploymentDescriptorNode.java}
	\sourcesnippet{526-526}{\ref{C13} violation at \codeline{526}}{Code/DeploymentDescriptorNode.java}
	\sourcesnippet{533-533}{\ref{C13} violation at \codeline{533}}{Code/DeploymentDescriptorNode.java}
	\item \ref{C14} \codeline{517}. The string concatenation statement length inside the function call exceeds 120 characters, consider wrapping it at 2\textsuperscript{nd} ``\texttt{+}''.
	\sourcesnippet{517-517}{\ref{C14} violation at \codeline{517}}{Code/DeploymentDescriptorNode.java}
	\item \ref{C31} \codeline{522}, \codeline{527} and \codeline{534}. These statements doesn't properly check if \texttt{value} is not \texttt{null} before use it.
	\sourcesnippet{522-522}{\ref{C31} violation at \codeline{522}}{Code/DeploymentDescriptorNode.java}
	\sourcesnippet{527-527}{\ref{C31} violation at \codeline{527}}{Code/DeploymentDescriptorNode.java}
	\sourcesnippet{534-534}{\ref{C31} violation at \codeline{534}}{Code/DeploymentDescriptorNode.java}
	\item \ref{C31} \codelines{521}{522}, \codelines{526}{527} and \codelines{533}{534}. These statements doesn't properly check if \texttt{target} is not \texttt{null} before use it.
	\sourcesnippet{521-522}{\ref{C31} violation at \codelines{521}{522}}{Code/DeploymentDescriptorNode.java}
	\sourcesnippet{526-527}{\ref{C31} violation at \codelines{526}{527}}{Code/DeploymentDescriptorNode.java}
	\sourcesnippet{533-534}{\ref{C31} violation at \codelines{533}{534}}{Code/DeploymentDescriptorNode.java}
	\item \ref{C31} \codeline{521}, \codeline{526} and \codeline{533}. These statements doesn't properly check if \texttt{methodName} is not \texttt{null} before use it.
	\sourcesnippet{521-521}{\ref{C31} violation at \codeline{521}}{Code/DeploymentDescriptorNode.java}
	\sourcesnippet{526-526}{\ref{C31} violation at \codeline{526}}{Code/DeploymentDescriptorNode.java}
	\sourcesnippet{533-533}{\ref{C31} violation at \codeline{533}}{Code/DeploymentDescriptorNode.java}
	\item \ref{C42} \codeline{517}. The message is misleading. It says that it's already in the target class' method but the invocation hasn't yet occurred. This could potentially lead to confusion if there are problems invoking the method at runtime using reflection, i.e. if one of the exceptions are caught.
	\sourcesnippet{517-517}{\ref{C42} violation at \codeline{517}}{Code/DeploymentDescriptorNode.java}
	\item \ref{C53} \codelines{524}{535}. Consider catching all the checked exceptions within the method rather than propagating the last one to the caller in case all the tries fail. If the error is unrecoverable a good solution is to throw a dedicated exception with a meaningful message containing the reason that raised the exceptional event. 
	\sourcesnippet{524-535}{\ref{C53} violation at \codelines{524}{535}}{Code/DeploymentDescriptorNode.java}
\end{enumerate}

\subsection{Issues related to method \texttt{writeSubDescriptors()}}

\subsection{Issues related to method \texttt{writeEjbReferenceDescriptors()}}