\subsection*{Naming Conventions}\begin{enumerate}[label=C\arabic*., ref=C\arabic*]
\item \checklistref All class names, interface names, method names, class variables, method variables, and constants used should have meaningful names and do what the name suggests.
\item \checklistref If one-character variables are used, they are used only for temporary ``throwaway'' variables, such as those used in for loops.
\item \checklistref Class names are nouns, in mixed case, with the first letter of each word in capitalized. Examples: \texttt{class Raster}; \texttt{class ImageSprite};
\item \checklistref Interface names should be capitalized like classes.
\item \checklistref Method names should be verbs, with the first letter of each addition word capitalized. Examples: \texttt{getBackground()}; \texttt{computeTemperature()}.
\item \checklistref Class variables, also called attributes, are mixed case, but might begin with an underscore (`\texttt{\_}') followed by a lowercase first letter. All the remaining words in the variable name have their first letter capitalized. Examples: \texttt{\_windowHeight}, \texttt{timeSeriesData}.
\item \checklistref Constants are declared using all uppercase with words separated by an underscore. Examples: \texttt{MIN\_WIDTH}; \texttt{MAX\_HEIGHT}.
\end{enumerate}

\subsection*{Indention}\begin{enumerate}[resume, label=C\arabic*., ref=C\arabic*]
\item \checklistref Three or four spaces are used for indentation and done so consistently.
\item \checklistref No tabs are used to indent.
\end{enumerate}

\subsection*{Braces}\begin{enumerate}[resume, label=C\arabic*., ref=C\arabic*]
\item \checklistref Consistent bracing style is used, either the preferred ``Allman'' style (first brace goes underneath the opening block) or the ``Kernighan and Ritchie'' style (first brace is on the same line of the instruction that opens the new block).
\item \checklistref All \texttt{if}, \texttt{while}, \texttt{do-while}, \texttt{try-catch}, and \texttt{for} statements that have only one statement to execute are surrounded by curly braces. Example:
avoid this:

% this can be done way better by using LISTINGS package

\begin{verbatim}
    if ( condition )
        doThis();
\end{verbatim}

instead do this:

\begin{verbatim}
    if ( condition ) 
    {
        doThis(); 
    }
\end{verbatim}

\end{enumerate}

\subsection*{File Organization}\begin{enumerate}[resume, label=C\arabic*., ref=C\arabic*]
\item \checklistref Blank lines and optional comments are used to separate sections (beginning comments, package/import statements, class/interface declarations which include class variable/attributes declarations, constructors, and methods).
\item \checklistref Where practical, line length does not exceed 80 characters.
\item \checklistref When line length must exceed 80 characters, it does NOT exceed 120 characters.
\end{enumerate}

\subsection*{Wrapping Lines}\begin{enumerate}[resume, label=C\arabic*., ref=C\arabic*]
\item \checklistref Line break occurs after a comma or an operator.
\item \checklistref Higher-level breaks are used.
\item \checklistref A new statement is aligned with the beginning of the expression at the same level as the previous line.
\end{enumerate}

\subsection*{Comments}\begin{enumerate}[resume, label=C\arabic*., ref=C\arabic*]
\item \checklistref Comments are used to adequately explain what the class, interface, methods, and blocks of code are doing.
\item \checklistref Commented out code contains a reason for being commented out and a date it can be removed from the source file if determined it is no longer needed.
\end{enumerate}

\subsection*{Java Source Files}\begin{enumerate}[resume, label=C\arabic*., ref=C\arabic*]
\item \checklistref Each Java source file contains a single public class or interface.
\item \checklistref The public class is the first class or interface in the file.
\item \checklistref Check that the external program interfaces are implemented consistently with what is described in the javadoc.
\item \checklistref Check that the javadoc is complete (i.e., it covers all classes and files part of the set of classes assigned to you).
\end{enumerate}

\subsection*{Package and Import Statements}\begin{enumerate}[resume, label=C\arabic*., ref=C\arabic*]
\item \checklistref If any package statements are needed, they should be the first non-comment statements. Import statements follow.
\end{enumerate}

\subsection*{Class and Interface Declarations}\begin{enumerate}[resume, label=C\arabic*., ref=C\arabic*]
\item \checklistref The class or interface declarations shall be in the following order:
	\begin{enumerate}
		\item class/interface documentation comment;
		\item class or interface statement;
		\item class/interface implementation comment, if necessary;
		\item class (static) variables;
		\begin{enumerate}
			\item first public class variables;
			\item next protected class variables;
			\item next package level (no access modifier);
			\item last private class variables.
		\end{enumerate}
		\item instance variables;
		\begin{enumerate}
			\item first public instance variables;
			\item next protected instance variables;
			\item next package level (no access modifier);
			\item last private instance variables.
		\end{enumerate}
		\item constructors;
		\item methods.
	\end{enumerate}
\item \checklistref Methods are grouped by functionality rather than by scope or accessibility.
\item \checklistref Check that the code is free of duplicates, long methods, big classes, breaking encapsulation, as well as if coupling and cohesion are adequate.
\end{enumerate}

\subsection*{Initialization and Declarations}\begin{enumerate}[resume, label=C\arabic*., ref=C\arabic*]
\item \checklistref Check that variables and class members are of the correct type. Check that they have the right visibility (public/private/protected).
\item \checklistref Check that variables are declared in the proper scope.
\item \checklistref Check that constructors are called when a new object is desired.
\item \checklistref Check that all object references are initialized before use.
\item \checklistref Variables are initialized where they are declared, unless dependent upon a computation.
\item \checklistref Declarations appear at the beginning of blocks (A block is any code surrounded by curly braces `\texttt{\{}' and `\texttt{\}}'). The exception is a variable can be declared in a \texttt{for} loop.
\end{enumerate}

\subsection*{Method Calls}\begin{enumerate}[resume, label=C\arabic*., ref=C\arabic*]
\item \checklistref Check that parameters are presented in the correct order.
\item \checklistref Check that the correct method is being called, or should it be a different method with a similar name.
\item \checklistref Check that method returned values are used properly.
\end{enumerate}

\subsection*{Arrays}\begin{enumerate}[resume, label=C\arabic*., ref=C\arabic*]
\item \checklistref Check that there are no off-by-one errors in array indexing (that is, all required array elements are correctly accessed through the index).
\item \checklistref Check that all array (or other collection) indexes have been prevented from going out-of-bounds.
\item \checklistref Check that constructors are called when a new array item is desired.
\end{enumerate}

\subsection*{Object Comparison}\begin{enumerate}[resume, label=C\arabic*., ref=C\arabic*]
\item \checklistref Check that all objects (including Strings) are compared with \texttt{equals} and not with \texttt{==}.
\end{enumerate}

\subsection*{Output Format}\begin{enumerate}[resume, label=C\arabic*., ref=C\arabic*]
\item \checklistref Check that displayed output is free of spelling and grammatical errors.
\item \checklistref Check that error messages are comprehensive and provide guidance as to how to correct the problem.
\item \checklistref Check that the output is formatted correctly in terms of line stepping and spacing.
\end{enumerate}

\subsection*{Computation, Comparisons and Assignments}\begin{enumerate}[resume, label=C\arabic*., ref=C\arabic*]
\item \checklistref Check that the implementation avoids ``brutish programming'': (see \url{http://users.csc.calpoly.edu/~jdalbey/SWE/CodeSmells/bonehead.html}). 
\item \checklistref Check order of computation/evaluation, operator precedence and parenthesizing.
\item \checklistref Check the liberal use of parenthesis is used to avoid operator precedence problems.
\item \checklistref Check that all denominators of a division are prevented from being zero.
\item \checklistref Check that integer arithmetic, especially division, are used appropriately to avoid causing unexpected truncation/rounding.
\item \checklistref Check that the comparison and Boolean operators are correct.
\item \checklistref Check throw-catch expressions, and check that the error condition is actually legitimate.
\item \checklistref Check that the code is free of any implicit type conversions.
\end{enumerate}

\subsection*{Exceptions}\begin{enumerate}[resume, label=C\arabic*., ref=C\arabic*]
\item \checklistref Check that the relevant exceptions are caught.
\item \checklistref Check that the appropriate action are taken for each catch block.
\end{enumerate}

\subsection*{Flow of Control}\begin{enumerate}[resume, label=C\arabic*., ref=C\arabic*]
\item \checklistref In a \texttt{switch} statement, check that all cases are addressed by break or return.
\item \checklistref Check that all switch statements have a default branch.
\item \checklistref Check that all loops are correctly formed, with the appropriate initialization, increment and termination expressions.
\end{enumerate}

\subsection*{Files}\begin{enumerate}[resume, label=C\arabic*., ref=C\arabic*]
\item \checklistref Check that all files are properly declared and opened.
\item \checklistref Check that all files are closed properly, even in the case of an error.
\item \checklistref Check that EOF conditions are detected and handled correctly.
\item \checklistref Check that all file exceptions are caught and dealt with accordingly.
\end{enumerate}


