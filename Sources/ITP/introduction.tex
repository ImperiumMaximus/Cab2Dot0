\newpage
\section{Introduction}
\subsection{Revision History}
\begin{table}[H]
	\centering
	\begin{tabular*}{\linewidth}{|p{0.1\linewidth}|p{0.13\linewidth}|p{0.3099\linewidth}|p{0.3499\linewidth}|}
		\hline
		\textbf{Version} & \textbf{Date}       & \textbf{Author(s)}         & \textbf{Summary}           \\ \hline
		0.1     & 01/08/2016 & Fioratto Raffaele & Document Creation \\ \hline
		0.2		& 01/09/2016 & Fioratto Raffaele & Added Introduction \\ \hline
		0.2.5	& 01/11/2016 & Fioratto Raffaele & Added Strategy (to be completed) \\ \hline 
		0.3		& 01/14/2016 & Fioratto Raffaele & Added some parts in section 4 \\ \hline
	\end{tabular*}
\end{table}
\break
\subsection{Purpose}
The purpose of an Integration Test Plan (ITP) is to describe the necessary tests to verify that all the components of a software are properly assembled and work together as expected. This document explain thoroughly the tests to be performed, the main criteria that led us to chose them, in which order they must be executed and their expected result. The Integration Test is one of the steps involved in the Verification \& Validation phase during Software Design and Development process. At this level, components of a software are threated like \textit{black-box}es (this is in contrast to unit testing in which they threat them as \textit{white-box}es) and test are crafted so that they are able to interact with other components. Each one of these tests have precise purpose and focuses on a particular result that must be achieved in order to consider the test passed.
\subsection{Scope}
The aim of the project is to create a new brand system that optimizes an 
existing taxi service.
The system will be capable of automatizing and/or simplifying certain 
processes during requests or reservations of taxis.
It will also guarantee a fair management of taxi queues.
New passengers can sign up for service inserting some basic information in order to use service's features as soon as possible.
A passenger can request a taxi using web service or through mobile
application after registration. The system will be able to localize precisely the position
of the passenger, determining taxis that are available near
him/her. The system will select a taxi and then it will forward the request to one of its driver.
Upon confirmation, the system will notify the customer about the successful completion of the operation and the ETA of the taxi. A passenger can also reserve a taxi by specifying time and date, but they have to do it with at least two hours in advance. Cancellation is also permitted. The system will actually process the request ten minutes before the time specified during the reservation in the same way as described previously.
\subsection{Definitions and Abbreviations}
\subsubsection{Definitions}
\subsubsection{Abbreviations}
\begin{itemize}
	\item ITP: \textbf{I}ntegration \textbf{T}est \textbf{P}lan
	\item DD: \textbf{D}esign \textbf{D}ocument
	\item RASD: \textbf{R}equirement \textbf{A}nalysis and \textbf{S}pecification \textbf{D}ocument
	\item IDE: \textbf{I}ntegrated \textbf{D}evelopment \textbf{E}nvironment
	\item JEE: \textbf{J}ava \textbf{E}nterprise \textbf{E}dition
	\item EJB: \textbf{E}nterprise \textbf{J}ava \textbf{B}ean
\end{itemize}
\subsection{Reference Documents}
\begin{itemize}
	\item myTaxiService Project Specification Document
	\item myTaxiService RASD
	\item myTaxiService DD
\end{itemize}