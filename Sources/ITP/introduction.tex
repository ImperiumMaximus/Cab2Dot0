\newpage
\section{Introduction}
\subsection{Revision History}
\begin{table}[H]
	\centering
	\begin{tabular*}{\linewidth}{|p{0.1\linewidth}|p{0.13\linewidth}|p{0.3099\linewidth}|p{0.3499\linewidth}|}
		\hline
		\textbf{Version} & \textbf{Date}       & \textbf{Author(s)}         & \textbf{Summary}           \\ \hline
		0.1     & 01/08/2016 & Fioratto Raffaele & Document Creation \\ \hline
		0.2		& 01/09/2016 & Fioratto Raffaele & Added Introduction \\ \hline
		0.2.5	& 01/11/2016 & Fioratto Raffaele & Added Strategy (to be completed) \\ \hline 
		0.3		& 01/15/2016 & Fioratto Raffaele & Completed Section 4 \\ \hline
		0.3.5	& 01/17/2016 & Fioratto Raffaele & Test Plan Skeleton \\ \hline
	\end{tabular*}
\end{table}
\break
\subsection{Purpose}
The purpose of an Integration Test Plan (ITP) is to describe necessary tests to verify that all the components of a software are properly assembled and work together. This document explains thoroughly tests to perform, why they have been chosen, in which order they must be executed and their expected result. The Integration Test is one of the steps involves in the Verification \& Validation phase during Software Design and Development process. At this level, components of a software are threated like \textit{black-box}es (this is in contrast to unit testing in which components are treated like \textit{white-box}es) and tests are crafted so that components are able to interact reciprocally. Each one of these tests has a precise purpose and focuses on a particular result that must be achieved in order to consider the test passed.
\subsection{Scope}
The aim of this project is to create a new brand system that optimizes an 
existing taxi service.
The system will be capable of automatizing and/or simplifying certain 
processes during requests or reservations of taxis.
It will also guarantee a fair management of taxi queues.
New passengers can sign up for services, inserting some basic information in order to use features as soon as possible.
A passenger can request a taxi using web service or mobile
application after registration. The system will be able to localize precisely his/her position, determining taxis that are available near
him/her. The system will select a taxi and then it will forward the request to one of its drivers.
Upon confirmation, the system will notify the customer about the successful completion of the operation and the ETA of the taxi. A passenger can also reserve a taxi by specifying time and date, but he/she has to do it with at least two hours in advance. Cancellation is also permitted. The system will actually process the request ten minutes before the time specified during the reservation in the same way as described previously.
\subsection{Definitions and Abbreviations}
\subsubsection{Definitions}
\subsubsection{Abbreviations}
\begin{itemize}
	\item ITP: \textbf{I}ntegration \textbf{T}est \textbf{P}lan
	\item DD: \textbf{D}esign \textbf{D}ocument
	\item RASD: \textbf{R}equirement \textbf{A}nalysis and \textbf{S}pecification \textbf{D}ocument
	\item IDE: \textbf{I}ntegrated \textbf{D}evelopment \textbf{E}nvironment
	\item JEE: \textbf{J}ava \textbf{E}nterprise \textbf{E}dition
	\item EJB: \textbf{E}nterprise \textbf{J}ava \textbf{B}ean
\end{itemize}
\subsection{Reference Documents}
\begin{itemize}
	\item myTaxiService Project Specification Document
	\item myTaxiService RASD
	\item myTaxiService DD
	\item \href{https://docs.jboss.org/author/display/ARQ/Reference+Guide}{Arquillian Documentation}\footnote{https://docs.jboss.org/author/display/ARQ/Reference+Guide}
	\item \href{http://junit.org/javadoc/latest/}{JUnit Documentation}\footnote{http://junit.org/javadoc/latest/}
	\item \href{http://mockito.github.io/mockito/docs/current/org/mockito/Mockito.html}{Mockito Documentation}\footnote{http://mockito.github.io/mockito/docs/current/org/mockito/Mockito.html}
\end{itemize}