\newpage
\section{Project Risks}
In this section risks for the project, their relevance in terms of reliability, availability of the system and associated recovery actions are defined.
For each risk, a probability of occurrence and relevance are given.
Concerning probability possible values are:
\begin{itemize}
	\item Low
	\item Moderate
	\item High
\end{itemize}
Whereas for relevance, it can be one of the following possible values in order:
\begin{itemize}
	\item Negligible
	\item Marginal
	\item Serious
	\item Critical
	\item Catastrophic
\end{itemize}
\begin{table}[H]
	\centering
	\begin{tabular}{p{0.6\linewidth}x{0.15\linewidth}x{0.15\linewidth}}
	\hline
	\textbf{Risk} & \textbf{Probability} & \textbf{Relevance} \\
	\hline
	Requirement volatility over time & Low & Serious \\
	Problem with team communication or subversive team members & Low & Serious \\
	Performance lack of external components (Databases, etc.) & Low & Marginal \\
	Delays over expected deadline & High & Critical \\ 
	Change of Regulations (e.g. Taxi policy, Car and Driver licenses, etc.) & Low & Serious \\
	Competitors & Moderate & Serious \\
	User acceptance (e.g. cumbersome and difficult to use interface, lack of responsiveness, etc.) & Moderate & Catastrophic \\
	Bankruptcy & Moderate & Catastrophic \\
	Integration Testing Failure & Low & Critical \\
	Downtime & Moderate & Critical \\
	Scalability issues & Moderate & Serious \\
	Security issues (e.g. loss of data caused by system hacking) & Moderate & Critical \\
	\hline
	\end{tabular}
	\caption{Summary of Project's risks}
\end{table} 
Next table illustrates possible actions in order to avoid or recover risks defined in the previous table, whenever possible, actions are designed to avoid risks proactively rather than reactively when they happen.
	\begin{longtable}{p{0.5\linewidth}x{0.5\linewidth}}
		\hline
		\textbf{Risk} & \textbf{Strategy} \\
		\hline
		\endhead
		Requirement volatility & This can't be proactively avoided in a definitive and fixed way, but can be mitigated designing reusable and extensible software. \\
		Team issues & Assigning tasks and responsibilities clearly to each project member. Try to maintain a fair and relaxed environment among members, in order to keep them highly motivated.\\
		Performance lack of external components & Consider either to investigate the reasons for poor performance or to buy another software component. \\
		Delays over expected deadline & Release a version of the software with missing features and then release an updated version with (hopefully) all the functionalities implemented. \\
		Change of Regulations & Perform a thoroughly study on current regulations in order to take them into account accordingly, however this may not be sufficient to avoid this risk completely.\\
		Competitors & Develop marketing plans (electronic advertising such as banners, publicity campaigns, etc.) in order to increase attractiveness of our product with respect to the others. \\
		User acceptance & Design carefully graphic interface to have a cutting-edge user experience, which encourages the acceptance.\\
		Bankruptcy & Carefully analyze costs and avoid delays, as well design a reliable system, so to reduce as much as possible needs to maintain the software (bug fixing, testing, patch development, etc.). \\
		Integration Testing Failure & Write carefully the DD and implement the component accordingly trying to follow as close as possible to the specification. Perform tests as early as possible. \\
		Downtime & This risk can be avoided by using some sort of redundancy at hardware or software level\\
		Scalability issues & Carefully design and implement the software to avoid this risk, accordingly test under heavy load conditions. \\
		Security issues & Use defensive programming as much as possible, never ``trust'' users, plan proper penetration tests to put under pressure the system, analyze results and if necessary, make changes.\\
		\hline
		\caption{Risks avoidance strategies}
	\end{longtable}