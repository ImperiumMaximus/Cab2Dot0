\newpage
\section{Introduction}
\subsection{Revision History}
\begin{table}[H]
	\centering
	\begin{tabular*}{\linewidth}{|p{0.1\linewidth}|p{0.13\linewidth}|p{0.3099\linewidth}|p{0.3499\linewidth}|}
		\hline
		\textbf{Version} & \textbf{Date}       & \textbf{Author(s)}         & \textbf{Summary}           \\ \hline
		0.1     & 01/21/2016 & Fioratto Raffaele & Document Creation \\ \hline
		0.2     & 01/22/2016 & Fioratto Raffaele & Initial Function Points and COCOMO sections implementation \\ \hline
		0.3		& 01/23/2016 & Fioratto Raffaele &
		Cost Estimation chapter completed \\ \hline
		0.4 	& 01/25/2016 & Longoni Nicol\`{o} & Project Task and Schedule chapter completed \\ \hline
		0.5 	& 01/26/2016 & Longoni Nicol\`{o} & Resource and Task chapter completed \\ \hline
		0.6		& 01/27/2016 & Longoni Nicol\`{o} & Introduction chapter completed \\ \hline
		0.7		& 01/28/2016 & Fioratto Raffaele & Risks written \\ \hline
		0.9		& 02/01/2016 & Fioratto Raffaele and Longoni Nicol\`{o} & Minor corrections \\ \hline	
		1.0		& 02/02/2016 & Fioratto Raffaele and Longoni Nicol\`{o} & Final Version \\ \hline
	\end{tabular*}
\end{table}
\break
\subsection{Purpose}
Project Plan document collects an investigation about all costs and risks of the overall project and not only the development phase.\\
This document includes various topics: first of all cost estimation which is a study about spendings in the project in terms of money, staff and other external efforts, besides development costs.\\
We continue with a Gantt diagram which is a working timetable that shows days assigned to complete specific tasks. After there is a schedule that explains the division of work between teammates.\\
We conclude with an analysis of risks, inclusive of probability and relevance for every possible risk listed. A way to avoid or solve these problematic situations is also presented. 
\subsection{Scope}
In this document we'll analyze all the project things related to cost estimation and efforts.\\
These evaluations are done through Function Point, a unit of measure that expresses the complexity of several activities related to software that are internal and external inputs, files and inquiries. With this data and the weight of every complexity level, we can calculate the UFP value to estimate the the overall cost of the project in FP.\\ 
Later we'll use COCOMO, an estimation model that estimates the effort, the lines of code written, driver scale and cost, project duration and people necessary to complete the work of a software.\\
Then a Gantt diagram will be presented to show, from the date when specifics have been given to us to project this software, how much time it is necessary to complete requested work, the subdivision of every project phase and the assignment of each part to a member of working group.\\
This document ends with a list of risks that could appear during planning matched with their probability to emerge and relevance. It is also presented a solution to prevent these inconveniences or in case this wasn't possible, a way to solve them.
\subsection{Definitions and Abbreviations}
\begin{itemize}
	\item PPD: \textbf{P}roject \textbf{P}lan \textbf{D}ocument
	\item ITPD: \textbf{I}ntegration \textbf{T}est \textbf{P}lan \textbf{D}ocument
	\item DD: \textbf{D}esign \textbf{D}ocument
	\item RASD: \textbf{R}equirement \textbf{A}nalysis and \textbf{S}pecification \textbf{D}ocument
	\item FP: \textbf{F}unction \textbf{P}oint
	\item UFP: \textbf{U}nadjusted \textbf{F}unction \textbf{P}oint
	\item COCOMO: \textbf{CO}nstructive \textbf{CO}st \textbf{MO}del
	\item Sys Admin: System Administrator
	\item SF: \textbf{S}cale \textbf{F}actor
\end{itemize}
\subsection{Reference Documents}
\begin{itemize}
	\item myTaxiService Project Specification Document
	\item myTaxiService RASD
	\item myTaxiService DD
	\item myTaxiService ITP
	\item \href{http://csse.usc.edu/csse/research/COCOMOII/cocomo2000.0/CII\_modelman2000.0.pdf}{COCOMO II Model Definition Manual}\footnote{Available at: \texttt{http://csse.usc.edu/csse/research/COCOMOII/cocomo2000.0/CII\_modelman2000.0.pdf}}
\end{itemize}