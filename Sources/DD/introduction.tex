\section{Introduction}
\subsection{Purpose}
This documentation represents the Design Document (DD). It describes completely the system in terms of components, analyzing their internal and external interactions (with the actors). It starts off with the description of the system given in the Requirements Analysis and Specification Document. In this document the system is treated like a white box, i.e. the document ``opens" it and then dissects it starting with a high level overview to a more detailed representation. This document is addressed to all developers and programmers who have to implement the actual software.
\subsection{Scope}
The aim of the project is to create a new brand system that optimizes an 
existing taxi service.
The system will be capable of automatizing and/or simplifying certain 
processes during requests or reservations of taxis.
It will also guarantee a fair management of taxi queues.
New passengers can sign up for service inserting some basic information in order to use service's features as soon as possible.
A passenger can request a taxi using web service or through mobile
application after registration. The system will be able to localize precisely the position
of the passenger, determining taxis that are available near
him/her. The system will select a taxi and then it will forward the request to one of its driver.
Upon confirmation, the system will notify the customer about the successful completion of the operation and the ETA of the taxi. A passenger can also reserve a taxi by specifying time and date, but they have to do it with at least two hours in advance. Cancellation is also permitted. The system will actually process the request ten minutes before the time specified during the reservation in the same way as described previously.
\subsection{Definitions, Acronyms, Abbreviations}
\subsubsection{Definitions}
\subsubsection{Acronyms}
\begin{itemize}
	\item DD: Design Document
	\item RASD: Requirement Analysis and Specification Document
	\item MVC: Model View Controller
	\item SOA: Service Oriented Application
	\item JEE: Java Enterprise Edition
	\item FSA: Finite State Automaton
	\item API: Application Programming Interface
	\item FSA: Finite State Automaton
	\item IDE: Integrated Development Environment
\end{itemize}
\subsubsection{Abbreviations}
\begin{itemize}
\item {[}Gn{]}: n-goal 
\item {[}Rn{]}: n-functional requirement
\item {[}Dn{]}: n-domain assumption
\item {[}An{]}: n-assumption
\end{itemize}
\subsection{Reference Documents}
\begin{itemize}
	\item myTaxiService Specification Document
	\item myTaxiService RASD
	\item DD Structure Template
\end{itemize}
\subsection{Document Structure}
\begin{itemize}
	\item Section 1: Introduction, it gives a description of this document, some basic information in order to clearly understand the subsequent sections.
	\item Section 2: Architectural Design, it describes the software to be designed starting with a high level representation and then dissecting it providing a more detailed analysis. Diagrams are provided to better clarify this part.
	\item Section 3: Algorithm Design: it gives a description of the main algorithms that are implemented in the software, pseudo-code or flow diagrams are provided to describe them in a clearer way.
	\item Section 4: User Interface Design: starting with mockups presented in the RASD, this part further explains interactions between actors and the system, and shows how it will look like.
	\item Section 5: Requirements Traceability: this part explains how requirements defined in the RASD map into the design elements that we shall define in this document.
	\item Section 6: References: this section provides a list of external documents and materials that allowed the composition of this documentation.
\end{itemize}
