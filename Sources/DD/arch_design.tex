\section{Architectural Design}
\subsection{Overview}
\label{sec:overview}
\subsection{High level components and their interaction}
\label{sec:high-level}
The following is a diagram that shows the composition of the system in terms of high level components. \newline
In the next section this component are expanded and further analyzed so eventually the overall system is completely described.
\begin{figure}[H]
	\centering
	\resizebox{6in}
	{!}{\input{HighLevelComponentDiagram.pdf_tex}}
	\caption{High level components diagram}
\end{figure}
As one can see in the diagram, the system is composed in three distinct levels:
\begin{itemize}
	\item GUI Logic
	\item Access Control and Ride Manager
	\item Database
\end{itemize}
This components can be easily mapped in the MVC architectural style (as later described in \nameref{sec:arch-styles-patterns} section).
Each components also is mapped into a specified tier among the server architecture as briefly described in the \nameref{sec:overview} section and further explained in the \nameref{sec:arch-styles-patterns} section.
The remaining component models all the possible actors, in the sense that it represents the physical things (a smartphone or a computer with the appropriate software installed on them) that actually interact with the system. \newline
The diagram clearly shows that there's a single macro-component that exposes a set of functionalities and methods that an actor can use. This macro-component is the only entry point available for all the actors, the reason for this choice is to preserve a simple architecture and also increase the cohesion and reusability of the component. \newline
This component is actually composed of a bunch of sub-components as described in the next section \nameref{sec:component-view}, each sub-component carry out specific functionalities for each actor defined in the RASD.
\subsection{Component View}
\label{sec:component-view}
\begin{figure}[H]
	\centering
	\resizebox{6in}
	{!}{\input{ComponentDiagramStripped.pdf_tex}}
\end{figure}
\subsection{Deployment View}
\begin{figure}[H]
	\centering
	\resizebox{6in}
	{!}{\input{DeploymentDiagram.pdf_tex}}
\end{figure}
\subsection{Runtime View}
\subsection{Component Interfaces}
In this section all the interfaces that connect each component in the component view are explained and analyzed.
\subsubsection{Passenger UI Logic Interface}
This interface provides the entry point to the system for the passenger. \newline
The following is a list of the methods provided by this interface, they are all pretty self-explanatory:
\begin{itemize}
	\item \texttt{showHome()}
	\item \texttt{showLogin()}
	\item \texttt{showSignUp()}
	\item \texttt{showRequestPage()}
	\item \texttt{showReservationPage()}
	\item \texttt{showRequestInformation()}
	\item \texttt{submitLoginCredentials()}
	\item \texttt{submitRegistrationCred()}
	\item \texttt{submitRequestData()}
	\item \texttt{submitReservationData()}
\end{itemize}
The interface is comprised as a series of methods that the passenger can invoke. Each \texttt{show*()} method maps to a specific web page or screen in the mobile application, the actual invocation takes place when a passenger loads in his/her browser or in the application one of the screens. Each page or screen has some sort of navigation methods so that it can call some of the other \texttt{show*()} methods present in this interface. \newline
The \texttt{submit*()} methods are called when the passenger wants to send data to system for instance when he/she has just completed the filling of a form, they validate the input and forward the request of the appropriate component in the system back-end, eventually they show a informative page or screen that notifies the user whether the operation completed or not, and in the latter case why. 
\subsubsection{Taxi Driver UI Logic Interface}
This interface provides the entry point to the system for the taxi driver.
The list of the methods is the following:
\begin{itemize}
	\item \texttt{showLogin()}
	\item \texttt{submitLoginCredentials()}
	\item \texttt{submitResponseForIncomingRequest()}
	\item \texttt{submitNewWorkingStatus()}
	\item \texttt{showNewIncomingRequest()}
\end{itemize}
The meaning and the functional aspects of these methods are more or less similar of what explained in the previous section, one point to note is that this time this methods are invoked only by the mobile application installed in the taxi driver's smartphone, so there is no web interface available for this specific actor.
\subsubsection{System Administrator UI Logic Interface}
This interface provides the entry point to the system for the administrator. \newline
As usual the methods list is the following:
\begin{itemize}
	\item \texttt{showLogin()}
	\item \texttt{showMainMenu()}
	\item \texttt{showAddNewDriver()}
	\item \texttt{showRemoveDriver()}
	\item \texttt{showModifyDriver()}
	\item \texttt{submitLoginCredentials()}
	\item \texttt{submitDriverInformations()}
	\item \texttt{submitDeleteRequest()}
\end{itemize}
One can notice that the name patterns for these methods are the same as in the previous sections.
\subsubsection{Access control Interface}
This interface supplies all the methods that enforce authentication and rights management in the system. \newline
The following is a list of these methods:
\begin{itemize}
	\item \texttt{login()}
	\item \texttt{logout()}
	\item \texttt{passengerSignUp()}
	\item \texttt{taxiDriverSignUp()}
	\item \texttt{modifyPassengerCredentials()}
\end{itemize}
\subsubsection{Ride manager Interface}
This interface contains the methods for creating and managing a single ride instance, either a request or a reservation. \newline
Methods list:
\begin{itemize}
	\item \texttt{createRequest()}
	\item \texttt{acceptRequest()}
	\item \texttt{declineRequest()}
	\item \texttt{createReservation()}
\end{itemize}
\subsubsection{Queue manager Interface}
This interface is comprised with methods that manages the availability of the taxis, functions for adding, or removing the taxis are supplied. \newline
List of methods:
\begin{itemize}
	\item \texttt{addTaxi()}
	\item \texttt{removeTaxi()}
	\item \texttt{changeTaxiStatus()}
\end{itemize}
\subsubsection{myTaxiService Database interface}
This interfaces is not directly implemented in the system-to-be, instead an existing third party library is used (JDBC) which already supplies its own set of interfaces so that the system can manipulate the database.
\subsubsection{Maps Service interface}
As with the previous section, also this interface is exposed by a third party component so that it exposes a series of methods that are suitable using in the system-to-be.
\subsection{Selected architectural styles and patterns}
\label{sec:arch-styles-patterns}

SOA, Client-Server, three tier, MVC, publisher-subscriber
\subsection{Other design decisions}