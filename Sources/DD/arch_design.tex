\section{Architectural Design}
\subsection{Overview}
\label{sec:overview}
The system is constituted by a three-tier architecture, in fact it's divided into GUI, application programs and database. The first one isn't only on users' devices, since the interface stays on users' smartphones and browsers, but the logic part is located on server side. On another server lays the application part where the system manages interactions between components and elaborates inputs coming from GUI. Database storage stays on a separated server from the logic one, it contains information about taxi drivers and their cabs, in addition to passenger personal details.  
\subsection{High level components and their interaction}
\label{sec:high-level}
The following diagram shows the composition of the system in terms of high level components. \newline
In the next section these components are explained and further analyzed to describe the overall system.
\begin{figure}[H]
	\centering
	\resizebox{6in}
	{!}{\input{HighLevelComponentDiagram.pdf_tex}}
	\caption{High level components diagram}
\end{figure}
As one can see in the diagram, the system is composed in three distinct levels:
\begin{itemize}
	\item GUI Logic
	\item Access Control and Ride Manager
	\item Database
\end{itemize}
These components can be easily mapped in the MVC architectural style (as later described in \nameref{sec:arch-styles-patterns} section).
Each component is also mapped in a specified tier among the server architecture as briefly described in the \nameref{sec:overview} section and further explained in the \nameref{sec:arch-styles-patterns} section.
The remaining component models all possible actors, in fact it represents physical elements (a smartphone or a computer with an appropriate software installed on them) that actually interact with the system. \newline
The diagram clearly shows that there is a single macro-component that exposes a set of functionalities and methods an actor can use. This macro-component is the only entry point available to all the actors, this choice has been made to preserve a simple architecture and also to increase the cohesion and reusability of the component itself. \newline
This component is actually composed of a bunch of sub-components as described in the next section \nameref{sec:component-view}, each sub-component carries out specific functionalities for each actor defined in the RASD.
\subsection{Component View}
\label{sec:component-view}
\begin{figure}[H]
	\centering
	\resizebox{6in}
	{!}{\input{ComponentDiagramStripped.pdf_tex}}
\end{figure}
\paragraph{Actors}
It's possible to see from this diagram a generic ``Actor'' component is now splitted in three different components, one for each actor, the main reason for this decision is that each actor accesses and uses a specific interface supplied in one of the three sub-components in the GUI Logic component. Methods that comprise such interfaces are described in the \nameref{sec:arch-styles-patterns} section.
\paragraph{GUI Logic}
The GUI Logic component hosts the interface logic suitable for using both a web browser and a mobile application, this component has the main role of collecting all requests from external devices of all actors, and then forward them to the right back-end component.
\paragraph{Access Control}
The Access Control component manages the authentication users, and their permissions, ensuring that only registered users can use certain functions of the system. \newline Since these functionalities are similar to all actors, this component is unique, and methods in its external interface are designed so that the component is aware of what type of actor is performing a specific action, the advantage of using such design is to increase the flexibility, and the reusability without sacrificing the simplicity of the overall system.
\paragraph{Ride Manager}
The Ride Manager component is in charge of managing all rides instances that are being served. This component creates, keeps track, terminates each ride managed by the system. Each instance can be seen as a FSA, and methods in component external interface implement a state transition in a automaton, this is further described in the \nameref{sec:arch-styles-patterns} section. \newline
This component manages both requests and reservations, in fact a reservation can be seen as a ``delayed'' request, thus with a little addition in algorithms, we can achieve both functionalities with little effort, maximizing reusability of this component.
\paragraph{Queue Manager}
This component manages all queues belonging to the city map, one queue for each zone. The component strictly cooperates with the Ride Manager, but its external interface is generic enough to abstract actual implementation of policies for managing queues. This allows two components to be separated and more flexible, whereas the architecture still remains simple and easy to understand, it is possible realize different implementations of the Queue Manager, for example a generic one for most of the zones, and a specific one for certain zones, for different reasons, for instance traffic load higher than normal, or a higher or a less probability to receive a request from a client in a specific zone. These policies can be considered during the implementation phase. \newline
The component also manages working state of Taxi drivers, it allows them to change between unavailable and available state then it removes or inserts them in an appropriate queue according to their location in the map (further described in the \nameref{sec:algo} section)
\paragraph{myTaxiService Database}
This database contains all information about registered users, both passengers and taxi drivers. Moreover it records all reservations and requests, active and terminated.
\paragraph{External Maps Services}
This external component is used to localize precisely where departure and arrival points are. Geolocalization systems are difficult to create without adequate instruments that can trace every route in the city and GPS systems. For this reason, we decided to entrust map services to an external company.
\paragraph{External Cars and Drivers database}
This database contains information about taxi drivers such as names, surnames, addresses, phone numbers, taxi driver license numbers and cab numbers. myTaxiService relies on this external data to control if all information about taxi drivers are legal.
\subsection{Deployment View}
The following picture is a deployment diagram for the myTaxiService system, each component showed in previous sections are mapped in a device and also communication protocols used between devices are showed.
\begin{figure}[H]
	\centering
	\resizebox{6in}
	{!}{\input{DeploymentDiagram.pdf_tex}}
\end{figure}
It can be noticed that each Actor corresponds to a single device. Regarding to mobile applications installed on devices, they will be developed using existing SDK frameworks which provide: already well defined interfaces and libraries to implement the rendering of the GUI, dealing with user inputs, and abstractions to communicate with the server. \newline
In database devices a MySQL implementation is used and exposed using JDBC which is comprised like a set of interfaces that are used in the Business Logic Tier and later discussed in the \nameref{sec:interfaces} section. \newline
JSON is used like a mean to encapsulate data exchanged between tiers, this brings the advantage to make independent the programming language used for implementing the overall system. 
\subsection{Runtime View}
This section presents some sequence diagrams that explain interaction between components defined in previous sections. This will further clarify how components work in the system as a whole, and also in which order should methods be called in order to accomplish a specific task. \newline
These diagrams don't take into account possible exceptions that can occur during the execution of a task, because of this complexity of the diagrams would increase, making more difficult their comprehension of them,. Also the notation used don't really implement a way for dealing with possible errors or exceptions.
\subsubsection{Passenger Login}
The following describes a sequence of actions performed when a visitor who is supposed to be already registered passenger, logs in the system. 
\begin{figure}[H]
	\small
	\centering
	\resizebox{6in}
	{!}{\input{LoginRuntime.pdf_tex}}
	\end{figure}
\break
\subsubsection{Passenger's Request}
The following describes the sequence of actions performed when a passenger wants to perform a request. 
\begin{figure}[H]
	\small
	\centering
	\resizebox{6in}
	{!}{\input{MakeRequest.pdf_tex}}
\end{figure}
\break
\subsubsection{Taxi Driver Registration}
The following describes the sequence of actions performed when a new Taxi Driver has just been hired and is registered into the system so that he/she can begin to work.
\begin{figure}[H]
	\small
	\centering
	\resizebox{6in}
	{!}{\input{DriverRegistration.pdf_tex}}
\end{figure}
\break
\subsubsection{Taxi Driver Availability status change}
The following describes the sequence of actions performed when a Taxi Driver is willing to change his/her working status.
\begin{figure}[H]
	\small
	\centering
	\resizebox{6in}
	{!}{\input{DriverStatusChange.pdf_tex}}
\end{figure}
\subsection{Component Interfaces}
\label{sec:interfaces}
In this section all interfaces that connect each component in the component view are explained and analyzed.
\subsubsection{Passenger UI Logic Interface}
This interface provides an entry point to the system for passengers. \newline
The following is a list of the methods provided by this interface, they are all pretty self-explanatory:
\begin{itemize}
	\item \texttt{showHome()}
	\item \texttt{showLogin()}
	\item \texttt{showSignUp()}
	\item \texttt{showRequestPage()}
	\item \texttt{showReservationPage()}
	\item \texttt{showRequestInformation()}
	\item \texttt{submitLoginCredentials()}
	\item \texttt{submitRegistrationCred()}
	\item \texttt{submitRequestData()}
	\item \texttt{submitReservationData()}
\end{itemize}
The interface comprised a series of methods that passengers can invoke. Each \texttt{show*()} method maps to a specific web page or screen in the mobile application, the actual invocation takes place when a passenger loads in his/her browser or in one screen of the application. Each page or screen has some sort of navigation methods so that it can call other \texttt{show*()} methods present in this interface. \newline
The \texttt{submit*()} methods are called when a passenger wants to send data to system, for instance when he/she has just completed the filling of a form, these methods validate inputs and forward the request to the appropriate component in the system back-end, eventually they show an informative page or a screen that notify the user whether the operation is completed or not, and in the latter case why. 
\subsubsection{Taxi Driver UI Logic Interface}
This interface provides the entry point to the system for the taxi driver.
The list of the methods is the following:
\begin{itemize}
	\item \texttt{showLogin()}
	\item \texttt{submitLoginCredentials()}
	\item \texttt{submitResponseForIncomingRequest()}
	\item \texttt{submitNewWorkingStatus()}
	\item \texttt{showNewIncomingRequest()}
\end{itemize}
Meaning and functional aspects of these methods are more or less similar to what has been explained in the previous section, one point to note is that this time, these methods are invoked only by the mobile application installed in the taxi driver's smartphone, so there is no web interface available to this specific actor.
\subsubsection{System Administrator UI Logic Interface}
This interface provides the entry point to the system for the administrator. \newline
List of methods:
\begin{itemize}
	\item \texttt{showLogin()}
	\item \texttt{showMainMenu()}
	\item \texttt{showAddNewDriver()}
	\item \texttt{showRemoveDriver()}
	\item \texttt{showModifyDriver()}
	\item \texttt{submitLoginCredentials()}
	\item \texttt{submitDriverInformation()}
	\item \texttt{submitDeleteRequest()}
\end{itemize}
It can notice that pattern names for these methods are the same of previous sections.
\subsubsection{Access Control Interface}
This interface supplies all the methods that enforce authentication and right management in the system. \newline
List of methods:
\begin{itemize}
	\item \texttt{login()}
	\item \texttt{logout()}
	\item \texttt{passengerSignUp()}
	\item \texttt{taxiDriverSignUp()}
	\item \texttt{passengerDelete()}
	\item \texttt{taxiDriverDelete()}
	\item \texttt{taxiDriverModifyCredentials()}
	\item \texttt{modifyPassengerCredentials()}
\end{itemize}
\subsubsection{Ride Manager Interface}
This interface contains methods to create and to manage a single ride instance, both requests and reservations. \newline
List of methods:
\begin{itemize}
	\item \texttt{createRequest()}
	\item \texttt{acceptRequest()}
	\item \texttt{declineRequest()}
	\item \texttt{createReservation()}
\end{itemize}
\subsubsection{Queue Manager Interface}
This interface comprises methods that manage the availability of taxis and supplies functions to add or to remove taxis. \newline
List of methods:
\begin{itemize}
	\item \texttt{addTaxi()}
	\item \texttt{removeTaxi()}
	\item \texttt{assignTaxi()}
	\item \texttt{changeTaxiStatus()}
\end{itemize}
\subsubsection{Databases interface}
This interface is not directly implemented in the system-to-be, instead an existing third party library (JDBC) is used, it supplies its own set of interfaces to the system that can manipulate the database.
\subsubsection{Maps API}
As with the previous section, also this interface is exposed to a third party component so that it can use a series of methods that are suitable in the system-to-be.
\subsection{Selected architectural styles and patterns}
\label{sec:arch-styles-patterns}
The following section lists and describes the architectural styles that represent feasible choices during development and deployment phases. All requirements specified in the RASD are taken into account with also design choices described in this document. 
\subsection{Architectural styles}
\paragraph{3-tier architecture}
As implicitly explained in the \nameref{sec:overview} section, this is perhaps the most important style, also because all component diagrams and also the deployment view are modelled against this architecture, from the latter diagram we can infer that tier map in this way:
\begin{itemize}
	\item GUI Logic to Presentation tier 
	\item Ride Manager, Access Control, Queue Manager, Maps to Logic tier
	\item Databases to the Data tier
\end{itemize}
Advantages coming from use of this syle is that it achieves a high level of modularity and flexibility of the overall system, also each tier can be designed and dimensioned based on the computational and power needed for it, moreover, as described in non-functional requirements in the RASD, each component is isolated with the respect to others to increase security in each of them and also in the overall system. One possible drawback is the increasing complexity of interfaces to design, in fact every tier has to communicate with others, also each user interaction corresponds to many API method calls within the system, which in turn trigger some further processes of exchanged data between tiers. This can produce a loss of performances and raise response times, but this is mitigated by the fact that our interfaces are simple enough and also require little of data to process when are passed from one tier to another.
\paragraph{SOA}
This style is used by the system for both services that are offered to ``Actors'' and third party services that the system uses to achieve its goals. Thus the system acts both as a service provider and as a service requester. \newline
Services provided are of course the ones that derive from goals already described in the RASD (request and reservation of taxis), plus some accessory functionalities that we also described in the RASD (login, signup, etc.).\newline
Requested services are the Maps Service and the external car and driver database that are needed in order to actually implement services offered by this system, both are already been defined, implemented and documented, in fact this document doesn't specify them in details but we assume it can be easily integrated in the system using the already existing documentation. 
\paragraph{Client-Server architecture}
This is tightly coupled to both 3-tier and SOA styles described previously. This is one of the most used and common style.
\subsection{Design patterns}
\paragraph{MVC}
\paragraph{Publisher-subscriber}
\paragraph{Fa\c{c}ade}
This pattern allows to users to communicate with applicative logic in an easy way, in fact they don't have to work with various components, but only with the fa\c{c}ade, it is up to this last one to communicate with other parts of the system.
\paragraph{Singleton}
This pattern allows to create one and only one instance of a determined object. For example it is used when a new registered user is created, in fact this pattern forbids to duplicate users already present in database. 
\paragraph{State pattern}
This pattern allows to change an object behavior during runtime. Here it is used to change ride states when a request or a reservation wait to be accepted, are working progress or are terminated.
\subsection{Other design decisions}
As already discussed in the RASD and showed in this document, we decided that the overall logic of the service will be implemented in Java using the JEE technology. This brings the main advantage to use a single programming language and a set of libraries on the server side that are able to deal with View, Controller and also Model, providing some techniques that are capable of abstracting the database, using Object Oriented Programming Paradigms. \newline
The View will be implemented using the Java Server Faces technology exposing methods in the GUI Logic presented in previous sections. The Controller will work tightly coupled with the rest of components again using Java and JEE, thus enabling a simpler and a coherent way to establish a communication between each tier. \newline Moreover, developers can use an IDE to implement actual code and to enhance features of the IDE itself, such as automated code inspection and testing (using frameworks like JUnit), to allow a more stable and bug-free code since the very first release. 
 